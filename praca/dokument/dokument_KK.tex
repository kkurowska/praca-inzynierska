\documentclass[twoside,a4paper]{book}

\usepackage{polski}
\usepackage[utf8]{inputenc}
\usepackage{amsmath,amssymb,amsthm}
\usepackage{xcolor}


\begin{document}
	
\chapter{Wstęp}
Porównanie rozkładu dwumianowego z rozkładem hipergeometrycznym.

Już od dawna wszystkim \cite{Edgeworth1918} znany jest rozkład dwumianowy, inaczej Bernoulliego. Modeluje on sytuację, gdy mamy $n$ prób i prawdopodobieństwo sukcesu $p$. Wtedy zmienna z rozkładu dwumianowego to liczba sukcesu w $n$ próbach 

\bibliographystyle{unsrt}
\bibliography{dokument_KK}

\end{document}
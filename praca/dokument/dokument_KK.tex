\documentclass[twoside,a4paper,12pt]{mwbk}

\usepackage{polski}
\usepackage[utf8]{inputenc}
%\usepackage{times}
\usepackage{amsmath,amssymb,amsthm}
\usepackage{xcolor}
\usepackage[final]{pdfpages}
\usepackage{graphicx}
%\usepackage[nottoc]{tocbibind}
\usepackage{caption}
\usepackage{subcaption}
\captionsetup{compatibility=false}
\usepackage{dsfont}
\usepackage{float}
%\usepackage{geometry}
%\newgeometry{tmargin=2.5cm, bmargin=2.5cm, lmargin=2.5cm, rmargin=2.5cm}

%\numberwithin{equation}{section}
%\numberwithin{figure}{section}
\renewcommand{\thefigure}{\thechapter.\arabic{figure}}

%\newcommand*{\doi}[1]{\href{http://dx.doi.org/#1}{doi: #1}}
%\newcommand*{\MR}[1]{\href{http://www.ams.org/mathscinet-getitem?mr=#1&return=pdf}{MR #1}}
%\newcommand*{\ZBL}[1]{\href{http://www.zentralblatt-math.org/zmath/en/advanced/?q=an:#1&format=complete}{Zbl #1}}


\newcommand{\1}[1]{\mathds{1}\left(#1\right)}

\newenvironment{diagrams}[2]{\begin{figure}[H]
	%\centering
	\begin{subfigure}{.5\textwidth}
		\centering
		\includegraphics[width=.9\textwidth]{wykresy/#1.png}
		\caption{}
		\label{#1}
	\end{subfigure}
	\begin{subfigure}{.5\textwidth}
		\centering
		\includegraphics[width=.9\textwidth]{wykresy/#2.png}
		\caption{}
		\label{#2}
	\end{subfigure}
	}
	{
	\end{figure}
	}

\newenvironment{subdiagrams}[2]{
		\begin{subfigure}{.5\textwidth}
			\centering
			\includegraphics[width=.9\textwidth]{wykresy/#1.png}
			\caption{}
			\label{#1}
		\end{subfigure}
		\begin{subfigure}{.5\textwidth}
			\centering
			\includegraphics[width=.9\textwidth]{wykresy/#2.png}
			\caption{}
			\label{#2}
		\end{subfigure}
	} {}


\begin{document}
	
\begin{titlepage}
	\includepdf{StronaTytulowa.pdf}
	\includepdf{StronaTytulowa_ang.pdf}
\end{titlepage}


\tableofcontents

\chapter*{Wstęp}
Początki teorii rachunku prawdopodobieństwa i~statystyki sięgają XVI~wieku. Na początku były to analizy rzutu kostką, czy prawdopodobieństwa błędów pomiarowych. Już w~XVII Blaise Pascal sformułował i~dowiódł własności trójkąta arytmetycznego oraz użył pojęcia kombinacji. Na początku XVIII~wieku opublikowane zostały prace Jacoba Bernoullego, w~których zawarł wiele swoich tez na temat prawdopodobieństwa. Przez te kilka wieków teoria rachunku prawdopodobieństwa i~statystyki znacząco się wzbogaciła i~rozwinęła. Rozpoczęto rozważania na temat estymacji i~testowania hipotez, które w~naszych czasach, są zasadniczą domeną statystyki.

W przypadku dyskretnym najczęściej testowane są proporcje populacji. Chcemy się przekonać czy dana próbka ma jakąś konkretną proporcję, albo czy dwie próbki mają tę samą proporcję. Znana jest powszechnie teoria dotycząca testowania hipotez, gdy populacja jest nieskończona, a~raczej na tyle duża, że możemy ją w przybliżeniu uznać za nieskończoną. Wtedy schemat próbkowania jest opisany przez rozkład Bernoullego. Jednak przypadek nieskończonej populacji nie wyczerpuje tematu testowania proporcji. Gdy populacja jest bardzo mała albo, gdy próbka jest niewiele mniejsza od całej populacji, schemat próbkowania opiera się o rozkład hipergeometryczny. 

Myślę, że warto zająć się teorią testowania hipotez dla skończonej populacji. W określonych przypadkach rozkład hipergeometryczny daje dużo dokładniejszą informację o badanym przypadku niż przybliżenie rozkładem dwumianowym. Ponadto zastosowanie tego typu testów ma duże znaczenie w~medycynie, gdzie często rozważane populacje mają na tyle wyspecjalizowane cechy, że są uważane za małe. \textit{Uzupełnić opis co się znajduje i gdzie!!!} W~kolejnym rozdziale znajduje się opis dwóch testów opartych o~rozkład hipergeometryczny oraz analiza prawdopodobieństwa błędu I~rodzaju i~mocy testu dla nich. Następnie wykonane jest porównanie tych testów z~testem wykorzystującym rozkładem Bernoullego.
\chapter{Schemat pobierania obserwacji}
\label{r1}
Gdy rozważamy skończoną populację, należy dokładnie przyjrzeć się schematowi pobierania obserwacji. Jakimi rozkładami są opisane próbki oraz jakie są własności tych rozkładów. Wyznaczymy także przedziały ufności dla proporcji~$p$ w~obu rozkładach. Przy czym, proporcja to stosunek wartości obserwacji do wielkości próbki, gdzie wartość obserwacji to liczba osobników z~badaną cechą w~próbce.


\section{Nieskończona populacja}
Gdy populacja jest bardzo duża, możemy traktować ją jako nieskończoną. Wobec tego losowanie kolejnych elementów próbki jest niezależne, czyli jest to losowanie ze zwracaniem. Zakładamy, że pobierane obserwacje pochodzą z~rozkładu Bernoulliego $\mathcal{B}(n,p)$, gdzie $n$~to rozmiar próby z~nieskończonej populacji, a~$p$~to proporcja zdarzeń sprzyjających w~populacji. Funkcja prawdopodobieństwa zmiennej losowej $X$ z~rozkładu dwumianowego jest równa
\begin{equation}
b(k;n,p) = P(X=k) = \binom{n}{k} p^k (1-p)^{n-k},\ 0\leq k\leq n.
\end{equation}

\section{Skończona populacja}
\label{r1:skonczonapopulacja}
Kiedy rozważamy populację skończoną, kolejne elementy próbki są losowane bez zwracania. Oznacza to, że prawdopodobieństwo sukcesu zmienia się w~trakcie pobierania elementów obserwacji w~zależności od wyboru poprzednich. W~takiej sytuacji próbka pochodzi z~rozkładu hipergeometrycznego $\mathcal{H}(n,M,N)$. Przy czym $n$~jest rozmiarem próbki, $M$~ilością osobników w~populacji z~daną cechą, a~$N$~rozmiarem populacji. Zmienna losowa~$X$ z~rozkładu hipergeometrycznego ma funkcję prawdopodobieństwa określoną wzorem
\begin{equation}
\label{hg}
h(k;n,M,N) = P(X=k) = \frac{\binom{M}{k} \binom{N-M}{n-k}}{\binom{N}{n}},\ L\leq k\leq U,
\end{equation}
gdzie
\begin{equation}
\label{ograniczenia}
L=\max\{0,M-N+n\},\quad U=\min\{n,M\}.
\end{equation}

Zauważmy, że wzór (\ref{hg}) ma prostą interpretację. Klasyczna definicja prawdopodobieństwa określa szansę zajścia zdarzenia~$A$ jako iloraz liczby zdarzeń elementarnych w~$A$ przez liczbę zdarzeń elementarnych w~$\Omega$, czyli $P(A) = |A|/|\Omega|$. W~rozważanym przypadku zdarzeniu~$A$ odpowiada sytuacja, w~której próbka będzie zawierać $k$~osobników z~daną cechą. Zatem ilość zdarzeń elementarnych w~$A$ to iloczyn dwóch kombinacji bez powtórzeń. Mamy: {\small $\binom{M}{k}$} możliwych wyborów $k$ osobników z $M$ posiadających daną cechę oraz {\small $\binom{N-M}{n-k}$} możliwych wyborów $n-k$ osobników z $N-M$ nieposiadających tej cechy. Wobec tego
\begin{equation}
|A| = \binom{M}{k} \binom{N-M}{n-k}.
\end{equation}
Natomiast zbiór wszystkich zdarzeń elementarnych w~$\Omega$ to wybór losowej próbki $n$~osobników z~$N$\dywiz elementowej populacji, wobec tego
\begin{equation}
|\Omega| = \binom{N}{n}.
\end{equation}
Ograniczenia nałożone na~$k$ są również naturalne. Dolne ograniczenie~$L$ jest równe maksimum z~$0$ i~$M-N+n$. Będzie ono niezerowe, gdy $M-N+n>0$. Przekształcając nierówność, otrzymujemy $n>N-M$, czyli rozmiar próbki jest większy od liczby osobników bez badanej cechy w~populacji. W~rezultacie mamy pewność, że w~próbce będzie przynajmniej tyle osobników z~daną cechą, ile wynosi różnica $n-(N-M)$. Ograniczenie górne jest równe minimum z~$n$~i~$M$, co wynika z~faktu, że nie może być więcej osób w~próbce z~daną cechą niż w~całej populacji. Analiza wzoru (\ref{hg}) pokazuje, że rozkład hipergeometryczny jest ściśle związany z~rozmiarem populacji.

\section{Porównanie rozkładów}

Rozkład hipergeometryczny daje bardzo podobne wyniki do rozkładu Bernoulliego, gdy populacja jest duża, albo~próbka stosunkowo mała. Natomiast przy małej populacji i~dużej próbce różnica między tymi rozkładami jest znaczna.

Rozważmy to na medycznym przykładzie. Załóżmy, że jest grupa $20$ osób, które są chore na jakąś bardzo rzadką chorobę oraz że $25\%$ z~nich ma szanse na wyzdrowienie. Chcemy dowiedzieć się, ile osób spośród przebadanych może wyzdrowieć. Weźmy $3$~różne próbki o~wielkościach~$n$ równych odpowiednio $10$, $17$ oraz $20$. Możemy tę sytuację opisać za pomocą rozkładu Bernoulliego, wtedy badana zmienna losowa jest z~rozkładu $\mathcal{B}(n,0.25)$. Drugim sposobem jest rozkład hipergeometryczny $\mathcal{H}(n,5,20)$. Rysunki \ref{pmf10}\ppauza\ref{pmf20} na stronie~\pageref{pmf10} przedstawiają funkcję prawdopodobieństwa dla wymienionych przypadków.

Przeanalizujmy jakie wartości mogą przyjmować zmienne losowe z~obu rozkładów. W~rozkładzie dwumianowym zmienna w~każdym z~przypadków przyjmuje wartości od $0$ do $n$. Na przykład na wykresie \ref{b_pmf10} argumenty funkcji to zbiór $\{0,1,\ldots,9,10\}$. 
Mimo że w populacji maksymalnie może wyzdrowieć $5$ osób, dla rozkładu Bernoulliego prawdopodobieństwo $X=6$ wynosi $0.09$. Przedstawiona rozbieżność wynika z~założenia, że pobieranie próbki to losowanie ze zwracaniem, czyli w~próbce mogą znaleźć się dwie te same osoby. Natomiast w~rozkładzie hipergeometrycznym zmienną losową ograniczają $L$ i~$U$, zdefiniowane wzorem (\ref{ograniczenia}), które uwzględniają wielkość próbki. Porównując, na wykresie \ref{hg_pmf10} zmienna losowa nie przyjmuje wartości większej niż $5$. Dodatkowo na wykresie \ref{hg_pmf17} najmniejszą wartością jest $2$, ponieważ populacja zawiera $15$ osób, które nie wyzdrowieją, zatem w~$17$\dywiz osobowej próbce jest pewne, że przynajmniej dwie będą zdrowe. Kolejne losowania są od siebie zależne, więc nie możemy drugi raz wylosować tej samej osoby.

\begin{diagrams}{b_pmf10}{hg_pmf10}
	\caption{Funkcje prawdopodobieństwa $b(k;10,0.25)$ oraz $h(k;10,5,20)$}
	\label{pmf10}
\end{diagrams}

\begin{diagrams}{b_pmf17}{hg_pmf17}
	\caption{Funkcje prawdopodobieństwa $b(k;17,0.25)$ oraz $h(k;17,5,20)$}
	\label{pmf17}
\end{diagrams}

\begin{diagrams}{b_pmf20}{hg_pmf20}
	\caption{Funkcje prawdopodobieństwa $b(k;20,0.25)$ oraz $h(k;20,5,20)$}
	\label{pmf20}
\end{diagrams}

Różnica między funkcjami prawdopodobieństwa obu rozkładów rośnie wraz ze wzrostem próbki. Skrajny przypadek przedstawia rysunek \ref{pmf20}, na którym próbka równa się populacji. Wykres \ref{hg_pmf20} idealnie obrazuje przypadek $n=N$. Pewne jest to, że w~$20$ osobach będzie dokładnie $5$, które mogą wyzdrowieć. Natomiast na wykresie \ref{b_pmf20} prawdopodobieństwo, że $X=5$ wynosi jedynie $0.2$, ponieważ funkcja prawdopodobieństwa dla rozkładu dwumianowego nie uwzględnia wielkości populacji. Warto również zaznaczyć, że im większa próbka, tym $P(X=5)$ dla rozkładu Bernoulliego jest coraz mniejsze, ponieważ mamy więcej elementów w~obserwacji. Tymczasem dla rozkładu hipergeometrycznego jest odwrotnie, funkcja prawdopodobieństwa dla $k=5$ rośnie, aż w~końcu osiąga wartość $1$. Gdy przebadamy więcej osobników, wzrasta nasza wiedza o~próbce oraz jest bardziej prawdopodobne to, że znajdzie się w~niej aż pięcioro zdrowych pacjentów.

Spójrzmy, jak wyglądają średnia i~wariancja dla rozważanych rozkładów.
Gdy $X\sim\mathcal{B}(n,p)$, to
\begin{equation}
E(X)=np,\quad Var(X)=np(1-p).
\end{equation}
Podczas gdy $X\sim\mathcal{H}(n,M,N)$, to
\begin{equation}
E(X)=n\frac{M}{N},\quad Var(X)=n\frac{M}{N}\left(1-\frac{M}{N}\right)\frac{N-n}{N-1}.
\end{equation}
Parametr $p$ rozkładu dwumianowego odpowiada ilorazowi parametrów $M/N$ w~rozkładzie hipergeometrycznym. Wobec tego średnie obu rozkładów są sobie równe, ale wariancje różni dodatkowy składnik w~rozkładzie hipergeometrycznym $(N-n)/(N-1)$. Przeanalizujmy, jak ten czynnik wpływa na zróżnicowanie rozkładów. Rysunek \ref{var} na stronie~\pageref{var} przedstawia wykresy wariancji dla rozważanego przykładu w~zależności od wielkości obserwacji. W~przypadku rozkładu dwumianowego wariancja stale rośnie wraz ze wzrostem próbki. Ostatecznie, gdy $n=20$ jest ona największa. Jednakże, gdy przebadamy całą populację, oczekiwanym rezultatem jest wartość zerowa wariancji, ponieważ nie ma wtedy losowości. Taki wynik daje nam wykres~\ref{hg_var} na stronie~\pageref{hg_var}. Funkcja na początku rośnie, ale gdy wielkość próbki przekroczy połowę rozmiaru populacji, wariancja zaczyna maleć aż do zera. Odzwierciedla to fakt, że gdy coraz więcej wiemy o~populacji, losowość uzyskanych wyników maleje.


\begin{figure}[h]
	\begin{subdiagrams}{b_var}{hg_var}
	\end{subdiagrams}

	\caption{Wariancja rozkładów Bernoulliego i~hipergeometrycznego w~zależności od rozmiaru próbki}
	\label{var}
\end{figure}

Rozważmy także, jak wyglądają przedziały ufności dla zmiennych losowych z~obu rozkładów. Załóżmy, że unormowana zmienna losowa pochodzi z~rozkładu standardowego normalnego, czyli
\begin{equation}
\frac{X-E(X)}{\sqrt{Var(X)}}\sim \mathcal{N}(0,1).
\end{equation}
Wobec tego dla $X\sim \mathcal{B}(n,p)$, jeśli aproksymujemy go rozkładem normalnym, przedział ufności parametru~$p$ jest w~przybliżeniu równy
\begin{equation}
\left[\, \hat{p}-z_{1-\alpha/2}\sqrt{\frac{\hat{p}(1-\hat{p})}{n}}\, , \,  \hat{p}+z_{1-\alpha/2}\sqrt{\frac{\hat{p}(1-\hat{p})}{n}}\, \right],
\end{equation}
gdzie $z_{1-\alpha/2}$ to kwantyl rozkładu normalnego rzędu~$1-\alpha/2$.
Natomiast dla $X\sim\mathcal{H}(n,M,N)$ przedział ufności proporcji $p=M/N$ to w~przybliżeniu
\begin{equation}
\left[\, \hat{p}-z_{1-\alpha/2} \sqrt{\frac{\hat{p}(1-\hat{p})(N-n)}{n(N-1)}}\, , \, \hat{p} + z_{1-\alpha/2} \sqrt{\frac{\hat{p}(1-\hat{p})(N-n)}{n(N-1)}}\, \right].
\end{equation}
W obu przypadkach estymator $\hat{p}=k/n$, gdzie $k$ to wartość obserwacji.

Kontynuując rozważany przykład, załóżmy teraz, że nie wiemy, ile osób może wyzdrowieć w~całej populacji, ale mamy wyniki badań. W~grupie $10$~osób wyzdrowiały $3$. Chcemy się dowiedzieć, ile osób może wyzdrowieć wśród wszystkich $20$. Podsumowując, znamy parametry $N=20$, $n=10$ i~wartość obserwacji $k=3$, zatem możemy wyznaczyć przedziały ufności dla parametru $p$ w~obu rozkładach. Dla rozkładu dwumianowego
\begin{equation}
p\in \left[\, 0.02 \, , \, 0.58\, \right].
\end{equation}
Natomiast dla rozkładu hipergeometrycznego
\begin{equation}
p\in \left[\, 0.09 \, , \, 0.5\, \right].
\end{equation}
Prawdziwy parametr $p=0.25$ faktycznie zawiera się w~tych przedziałach. Zauważmy, że przedział dla rozkładu Bernoulliego jest szerszy, ze względu na brak skończonej poprawki w~wariancji.





\chapter{Przedstawienie testów}
\label{r2}
W niniejszym rozdziale znajduje się opis trzech testów. Dwa testy ze skończoną poprawką wykorzystują rozkład hipergeometryczny, a~trzeci test, bez skończonej poprawki, opiera się o~rozkład dwumianowy. Następnie omówiony jest sposób liczenia mocy dla wymienionych testów.

\section{Sformułowanie problemu}
Załóżmy, że $X_1$ i~$X_2$ są niezależnymi zmiennymi losowymi. Zaobserwowane wartości $X_1$ i~$X_2$ oznaczmy odpowiednio $k_1$ i~$k_2$ oraz proporcje w~obserwacjach $p_1$ i~$p_2$. Będziemy testować
\begin{equation}
H_0{:}\ p_1=p_2\quad \text{przeciwko} \quad H_1{:}\ p_1\neq p_2,
\end{equation}
na podstawie wartości obserwacji i~znanych parametrów populacji.
Rozważmy unormowaną statystykę
\begin{equation}
Z_{X_1,X_2} = \frac{X_1/n_1-X_2/n_2}{\sqrt{V_{X_1,X_2}}},
\end{equation}
gdzie $V_{X_1,X_2}$ to estymator wariancji rozkładu zmiennej losowej $X_1/n_1-X_2/n_2$, pod warunkiem prawdziwości $H_0$, w~połączonej próbie. Jego wzór zależy od rozkładu, z~którego pochodzą zmienne losowe $X_1$ i~$X_2$.
Wartość statystyki $Z_{X_1,X_2}$ oznaczmy jako $Z_{k_1,k_2}$. Jest ona wyliczana według powyższych wzorów, poprzez zamienienie zmiennych losowych $X_1$ i~$X_2$ odpowiednio ich wartościami $k_1$ i~$k_2$.

\section{Testy ze skończoną poprawką}
\label{r2:skonczonapoprawka}
Jak już było wspomniane w~podrozdziale \ref{r1:skonczonapopulacja}, próbki w~przypadku skończonej populacji pochodzą z~rozkładu hipergeometrycznego. Zatem $X_1\sim \mathcal{H}(n_1,M_1,N_1)$, $X_2\sim \mathcal{H}(n_2,M_2,N_2)$ oraz proporcje są równe $p_1=M_1/N_1$, $p_2=M_2/N_2$. Znane parametry to rozmiary próbek $n_1$ i~$n_2$ i~wielkości populacji $N_1$ i~$N_2$.

W celu wyprowadzenia wariancji rozkładu $X_1/n_1-X_2/n_2$ pod warunkiem $p_1=p_2$ w~połączonej próbie, zapiszmy wariancję rozważanej zmiennej losowej w~łącznej próbie, korzystając z~własności wariancji oraz tego, że $Cov(X_1,X_2)=0$ z~niezależności $X_1$ i~$X_2$
\begin{equation}
\begin{split}
Var(X_1/n_1-X_2/n_2)=&Var(X_1/n_1) + Var(X_2/n_2) =\\
 =& Var(X_1)/n_1^2+Var(X_2)/n_2^2.
\end{split}
\end{equation}
Wariancje $X_1$ i~$X_2$ są równe
\begin{align}
Var(X_1)=n_1 p_1 (1-p_1)(N_1-n_1)/(N_1-1),\\
Var(X_2)=n_2 p_2 (1-p_2)(N_2-n_2)/(N_2-1).
\end{align}
Pamiętając, że zakładamy równość $p_1=p_2$ zastąpmy oba parametry jednym równym $p$. Po podstawieniu otrzymujemy
\begin{equation}
\begin{split}
V_{X_1,X_2} & = \frac{1}{n_1}p(1-p)\frac{N_1-n_1}{N_1-1} + \frac{1}{n_2}p(1-p)\frac{N_2-n_2}{N_2-1}= \\
&= p(1-p)\left(\frac{N_1-n_1}{n_1(N_1-1)}+\frac{N_2-n_2}{n_2(N_2-1)}\right),
\end{split}
\end{equation}
przy czym $p$ to proporcja liczby osobników z~daną cechą do całości populacji w~rozkładzie łącznym. Wobec czego $p=(X_1+X_2)/(n_1+n_2)$. Ostatecznie otrzymujemy
\begin{equation}
V_{X_1,X_2} = \left(\frac{N_1-n_1}{n_1(N_1-1)}+\frac{N_2-n_2}{n_2(N_2-1)}\right)\left(\frac{X_1+X_2}{n_1+n_2}\right)\left(1-\frac{X_1+X_2}{n_1+n_2}\right).
\end{equation}

\subsection{Test Z}
Test~Z jest oparty na centralnym twierdzeniu granicznym, według którego, rozważana statystyka $Z_{X_1,X_2}$, pod warunkiem prawdziwości $H_0$, jest w~przybliżeniu z~rozkładu standardowego normalnego $\mathcal{N}(0,1)$. Wtedy $p$-wartość wyraża się wzorem
\begin{equation}
P(|Z_{X_1,X_2}|\geq|Z_{k_1,k_2}|\ |H_0) \approx 2(1-\Phi(|Z_{k_1,k_2}|)),
\end{equation}
gdzie $\Phi$ oznacza dystrybuantę rozkładu $N(0,1)$. Test Z~odrzuca hipotezę zerową, gdy $p$-wartość jest mniejsza od poziomu istotności $\alpha$.

\subsection{Test E}
Test E opiera się o~rzeczywistą $p$-wartość, która, według artykułu K. Krishnamoorthy i~J. Thomson z~2002 roku, jest równa \cite{K.Krishnamoorthy2002}
\begin{equation}
\label{realpvalue}
\begin{split}
&P(|Z_{X_1,X_2}|\geq|Z_{k_1,k_2}|\ |H_0) = E_{X_1,X_2}(\1{|Z_{X_1,X_2}|\geq|Z_{k_1,k_2}|}\ |H_0) = \\
&= \sum_{x_1=L_1}^{U_1}\sum_{x_2=L_2}^{U_2} h(x_1;n_1,N_1p,N_1)h(x_2;n_2,N_2p,N_2)\1{|Z_{X_1,X_2}|\geq|Z_{k_1,k_2}|},
\end{split}
\end{equation}
gdzie $E_{X_1,X_2}$ to wartość oczekiwana łącznego rozkładu $(X_1,X_2)$, a~$p$ jest nieznaną wspólną proporcją pod warunkiem $H_0$. Nie jest możliwe policzenie $p$-wartości wprost ze wzoru (\ref{realpvalue}), ponieważ nie znamy parametru proporcji~$p$. Krishnamoorthy i~Thomson (2002) zaproponowali estymator $p$-wartości \cite{K.Krishnamoorthy2002}
\begin{equation}
\label{estpvalue}
\begin{split}
P(|Z_{X_1,X_2}|\geq|Z_{k_1,k_2}|\ |H_0) \approx& \\ =\sum_{x_1=L_{x_1}}^{U_{x_1}}\sum_{x_2=L_{x_2}}^{U_{x_2}} h(x_1;n_1,\hat{M_1},&N_1)h(x_2;n_2,\hat{M_2},N_2)\1{|Z_{x_1,x_2}|\geq|Z_{k_1,k_2}|},
\end{split}
\end{equation}
przy czym $\hat{p}=(k_1+k_2)/(n_1+n_2)$, $\hat{M_i}=[N_i\hat{p}]$ oraz  $L_{x_i}=\max\{0,\hat{M_i}-N_i+n_i\}$, $U_{x_i}=\min\{n_i,\hat{M_i}\}$, $i=1,2$. Test odrzuca $H_0$ wtedy, gdy $p$-wartość wyliczona według wzoru (\ref{estpvalue}) jest mniejsza od poziomu istotności $\alpha$.

\subsection{Moc testu}
Moc testu to prawdopodobieństwo odrzucenia hipotezy zerowej, gdy jest ona nieprawdziwa. Wobec tego, spośród testów na zadanym poziomie istotności, interesują nas te o~najwyższej mocy.

Moc obu testów można wyliczyć, korzystając z~funkcji prawdopodobieństwa rozkładu hipergeometrycznego. Dla testu~Z pod warunkiem hipotezy alternatywnej $H_1$ moc, zgodnie z~rozważaniami Krishnamoorthy i~Thomson (2002), jest równa \cite{K.Krishnamoorthy2002}
\begin{equation}
\label{powerZ}
\sum_{k_1=L_1}^{U_1}\sum_{k_2=L_2}^{U_2} h(k_1;n_1,M_1,N_1)h(k_2;n_2,M_2,N_2)\1{|Z_{k_1,k_2}|>z_{1-\alpha/2}},
\end{equation}
gdzie $L_i=\max\{0,M_i-N_i+n_i\}$ i~$U_i=\min\{n_i,M_i\}$, a~$z_{1-\alpha/2}$ oznacza kwantyl rozkładu normalnego standardowego rzędu $1-\alpha/2$.

Natomiast dla testu E, według Krishnamoorthy i~Thomson (2002), moc zdefiniowana jest następująco \cite{K.Krishnamoorthy2002}
\begin{equation}
\begin{split}
&\sum_{k_1=L_1}^{U_1}\sum_{k_2=L_2}^{U_2} h(k_1;n_1,M_1,N_1)h(k_2;n_2,M_2,N_2) \times \\
&\times \1{\sum_{x_1=L_{x_1}}^{U_{x_1}}\sum_{x_2=L_{x_2}}^{U_{x_2}} h(x_1;n_1,\hat{M_1},N_1)h(x_2;n_2,\hat{M_2},N_2) \1{|Z_{x_1,x_2}|\geq|Z_{k_1,k_2}|}\leq\alpha },
\end{split}
\end{equation}
gdzie parametry są takie same jak we wzorach (\ref{estpvalue}) i~(\ref{powerZ}).

\section{Test bez skończonej poprawki}
Dla testu bez poprawki na skończony rozmiar populacji, zamiast rozkładu hipergeometrycznego stosujemy dwumianowy, więc $X_1$ i~$X_2$ są niezależnymi zmiennymi losowymi o~rozkładzie Bernoulliego $X_1\sim \mathcal{B}(n_1,p_1)$, $X_2\sim \mathcal{B}(n_2,p_2)$. Znane parametry to rozmiary próbek $n_1$ i~$n_2$.

Wariancję rozkładu $X_1/n_1-X_2/n_2$, pod warunkiem $p_1=p_2$, w~łącznej próbie możemy wyprowadzić analogicznie jak w~podrozdziale \ref{r2:skonczonapoprawka}, wychodząc od wariancji rozważanej zmiennej losowej
\begin{equation}
Var(X_1/n_1-X_2/n_2) = Var(X_1)/n_1^2+Var(X_2)/n_2^2.
\end{equation}
Wariancje $X_1$ i~$X_2$ są równe
\begin{align}
Var(X_1)=n_1 p_1 (1-p_1),\\
Var(X_2)=n_2 p_2 (1-p_2).
\end{align}
Zastępując $p_1$ i~$p_2$ jednym parametrem równym $p$, otrzymujemy
\begin{equation}
V_{X_1,X_2} = p(1-p)/n_1 + p(1-p)/n_2 = p(1-p)(1/n_1+1/n_2)
\end{equation}
przy czym $p=(X_1+X_2)/(n_1+n_2)$.

\subsection{Test Zb}

Test~Zb jest, podobnie jak omówiony wcześniej test E, oparty o~estymator $p$-wartości, który, zgodnie z~artykułem Storer i~Kim z~1990 roku, jest równy \cite{Storer1990}
\begin{equation}
\begin{split}
P(|Z_{X_1,X_2}|&\geq|Z_{k_1,k_2}|\ |H_0) \approx \\
\approx & \sum_{x_1=0}^{n_1}\sum_{x_2=0}^{n_2} b(x_1;n_1,\hat{p_1})b(x_2;n_2,\hat{p_2})\1{|Z_{X_1,X_2}|\geq|Z_{k_1,k_2}|},
\end{split}
\end{equation}
gdzie $\hat{p}=(k_1+k_2)/(n_1+n_2)$.
Test odrzuca hipotezę zerową, gdy $p$-wartość jest mniejsza od poziomu istotności $\alpha$.

Moc testu wyraża się wzorem
\begin{equation}
\begin{split}
\sum_{k_1=0}^{n}\sum_{k_2=0}^{n}& b(k_1;n_1,p_1)b(k_2;n_2,p_2) \times \\
\times &\1{\sum_{x_1=0}^{n_1} \sum_{x_2=0}^{n_2} b(x_1;n_1,\hat{p_1}) b(x_2;n_2,\hat{p_2}) \1{|Z_{x_1,x_2}|\geq|Z_{k_1,k_2}|} \leq\alpha}.
\end{split}
\end{equation}


\chapter{Analiza testów}

Do porównania testów napisałam programy, które wyliczają prawdopodobieństwo błędu I rodzaju oraz moc testu dla różnych parametrów. 



\begin{figure}[!h]
	\begin{subdiagrams}{sizeZE_p_0_05.png}{sizeZE_p_0_05_n1_5.png}
	\end{subdiagrams}
	
	\begin{subdiagrams}{sizeZE_p_0_1.png}{sizeZE_p_0_1_n1_10.png}
	\end{subdiagrams}
	
	\begin{subdiagrams}{sizeZE_p_0_3.png}{sizeZE_p_0_3_n1_5.png}
	\end{subdiagrams}
	\caption{P-stwo błędu I rodzaju jako funkcja rozmiaru próbki $n$, przy zadanym poziomie istotności $\alpha=0.05$; $N_1=N_2=100$}
	\label{sizeZE_n}
\end{figure}


\begin{figure}[!h]
	\begin{subdiagrams}{sizeZE_n_10.png}{sizeZE_n1_10_n2_6.png}
	\end{subdiagrams}
	
	\begin{subdiagrams}{sizeZE_n_20.png}{sizeZE_n1_20_n2_5.png}
	\end{subdiagrams}
	
	\begin{subdiagrams}{sizeZE_n_30.png}{sizeZE_n1_30_n2_15.png}
	\end{subdiagrams}
	\caption{P-stwo błędu I rodzaju jako funkcja parametru $p=M_1/N_1=M_2/N_2$, przy zadanym poziomie istotności $\alpha=0.05$; $N_1=N_2=100$}
	\label{sizeZE_p}
\end{figure}

\begin{figure}[!h]
	\begin{subdiagrams}{powerZE_N1_30_N2_50_p_0_6.png}{powerZE_N_100_p_0_1.png}
	\end{subdiagrams}
	
	\begin{subdiagrams}{powerZE_N1_100_N2_200_p_0_1.png}{powerZE_N1_3000_N2_100_p_0_6.png}
	\end{subdiagrams}
	
	\caption{Moc testu jako funkcja rozmiaru próbki $n$, przy zadanym poziomie istotności $\alpha=0.05$}
	\label{powerZE_n}
\end{figure}
\chapter*{Podsumowanie}

W pracy został rozważony problem testowania hipotez i~estymacji w~sytuacji populacji skończonego rozmiaru. W~rozdziale \ref{r1} przedstawiono schematy pobierania obserwacji. W~sytuacji nieskończonej populacji schemat opiera się o~rozkład dwumianowy i~próbka jest losowana ze zwracaniem. Natomiast, gdy populacja jest skończona, pobieranie obserwacji modeluje rozkład hipergeometryczny, przy czym elementy próbki są losowane bez zwracania. Po analizie obu rozkładów na konkretnym przykładzie sformułowano wnioski, że rozważane rozkłady dają podobne wyniki, jeśli populacja jest duża albo próbka stosunkowo mała. Jednakże, gdy obserwacja stanowi znaczną część populacji, albo sama populacja jest mała, to różnica między rozkładami staje się bardzo widoczna. Dzieje się tak, ponieważ rozkład hipergeometryczny uwzględnia wielkość populacji, a~rozkład Bernoulliego nie uwzględnia takiej informacji.

Rozdział \ref{r3} zawiera analizę porównawczą testów opisanych w~rozdziale~\ref{r2}, na podstawie wykresów prawdopodobieństwa błędu I rodzaju i~mocy testu, dla różnych parametrów testowanych populacji. Porównanie dwóch testów ze skończoną poprawką wskazało na to, że test~Z nie utrzymuje poziomu istotności $\alpha$, ze względu na zastosowania aproksymacji rozkładem normalnym do statystyki testowej. Natomiast test~E nie wykracza znacząco powyżej poziomu istotności oraz jego moc jest niewiele mniejsza od mocy testu Z. Tym samym, test~E możemy uznać za dobry w~przypadku małej populacji.

W celu pokazania zasadności stosowania testu ze skończoną poprawką do małych populacji porównano także test~E z~testem bez skończonej poprawki. Na wykresach przedstawiających błąd I rodzaju możemy zauważyć, że test~Zb znacząco przekracza poziom istotności $\alpha$, szczególnie dla małych próbek. Oznacza to, że zbyt często odrzuca hipotezę zerową. Wobec czego stosowanie testu bez skończonej poprawki w~sytuacji małej populacji wiąże się z~dużymi błędami.

Problem skończonej populacji często występuje w~medycynie. Lekarze badają przypadki, w~których populacja jest mała, ze względu na konkretne cechy badanych osób. We Wrocławiu w~Ośrodku Badawczo-Rozwojowym przy Wojewódzkim Szpitalu Specjalistycznym jeden z~kardiologów przeprowadza badania na populacji dzieci z~chorym sercem po operacji. Chce on sprawdzić jakie czynniki wpływają na wystąpienie powikłań pooperacyjnych u pacjentów. Niestety nie można na obecną chwilę przeprowadzić analizy danych, ponieważ badanie nie zostało jeszcze ukończone. Jednakże, aby sprawdzić, który czynnik jest najbardziej powiązany z~powikłaniami, można zastosować opisany w~pracy test E. Powinien dać on najlepsze wyniki spośród trzech testów przedstawionych w~pracy.

\listoffigures
%\listoftables

\bibliographystyle{unsrt}
%\bibliographystyle{abbrv}
\bibliography{dokument_KK}
	
\end{document}
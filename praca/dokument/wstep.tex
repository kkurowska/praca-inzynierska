\chapter*{Wstęp}
Początki teorii rachunku prawdopodobieństwa i~statystyki sięgają XVI~wieku. Na początku były to analizy rzutu kostką, czy prawdopodobieństwa błędów pomiarowych. Już w~XVII Blaise Pascal sformułował i~dowiódł własności trójkąta arytmetycznego oraz użył pojęcia kombinacji. Na początku XVIII~wieku opublikowane zostały prace Jacoba Bernoullego, w~których zawarł wiele swoich tez na temat prawdopodobieństwa. Przez te kilka wieków teoria rachunku prawdopodobieństwa i~statystyki znacząco się wzbogaciła i~rozwinęła. Rozpoczęto rozważania na temat estymacji i~testowania hipotez, które w~naszych czasach, są zasadniczą domeną statystyki.

W przypadku dyskretnym najczęściej testowane są proporcje populacji. Chcemy się przekonać czy dana próbka ma jakąś konkretną proporcję, albo czy dwie próbki mają tę samą proporcję. Znana jest powszechnie teoria dotycząca testowania hipotez, gdy populacja jest nieskończona, a~raczej na tyle duża, że możemy ją w przybliżeniu uznać za nieskończoną. Wtedy schemat próbkowania jest opisany przez rozkład Bernoullego. Jednak przypadek nieskończonej populacji nie wyczerpuje tematu testowania proporcji. Gdy populacja jest bardzo mała albo, gdy próbka jest niewiele mniejsza od całej populacji, schemat próbkowania opiera się o rozkład hipergeometryczny. 

Myślę, że warto zająć się teorią testowania hipotez dla skończonej populacji. W określonych przypadkach rozkład hipergeometryczny daje dużo dokładniejszą informację o badanym przypadku niż przybliżenie rozkładem dwumianowym. Ponadto zastosowanie tego typu testów ma duże znaczenie w~medycynie, gdzie często rozważane populacje mają na tyle wyspecjalizowane cechy, że są uważane za małe. \textit{Uzupełnić opis co się znajduje i gdzie!!!} W~kolejnym rozdziale znajduje się opis dwóch testów opartych o~rozkład hipergeometryczny oraz analiza prawdopodobieństwa błędu I~rodzaju i~mocy testu dla nich. Następnie wykonane jest porównanie tych testów z~testem wykorzystującym rozkładem Bernoullego.
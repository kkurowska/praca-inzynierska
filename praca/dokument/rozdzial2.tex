\chapter{Przedstawienie testów}

W niniejszym rozdziale są przedstawione trzy testy, jeden oparty na rozkładzie dwumianowym i~dwa wykorzystujące rozkład hipergeometryczny. Jest także omówiony sposób liczenia dla nich mocy testu.

\section{Sformułowanie problemu}
Załóżmy, że $X_1$ i~$X_2$ są niezależnymi zmiennymi losowymi. Zaobserwowane wartości $X_1$ i~$X_2$ oznaczmy odpowiednio $k_1$ i~$k_2$ oraz proporcje w~obserwacjach $p_1$ i~$p_2$. Będę zajmować się testowaniem hipotez
\begin{equation}
H_0{:}\ p_1=p_2\quad \text{przeciwko} \quad H_1{:}\ p_1\neq p_2,
\end{equation}
w oparciu o~wartości obserwacji i~znane parametry populacji.
Rozważmy unormowaną statystykę
\begin{equation}
Z_{X_1,X_2} = \frac{X_1/n_1-X_2/n_2}{\sqrt{V_{X_1,X_2}}},
\end{equation}
gdzie $V_{X_1,X_2}$ to estymator wariancji rozkładu łącznego zmiennej losowej $X_1/n_1-X_2/n_2$ pod warunkiem prawdziwości $H_0$. Jego wzór zależy od rozkładu, z~którego pochodzą zmienne losowe $X_1$ i~$X_2$.
Wartość statystyki $Z_{X_1,X_2}$ oznaczmy jako $Z_{k_1,k_2}$. Jest ona wyliczana według powyższych wzorów, poprzez zamienienie $X_1$ i~$X_2$ odpowiednio wartościami obserwacji $k_1$ i~$k_2$.

\section{Testy ze skończoną poprawką}
\label{r2:skonczonapoprawka}
Jak już było wspomniane w~sekcji \ref{r1:skonczonapopulacja}, próbki w~przypadku skończonej populacji pochodzą z~rozkładu hipergeometrycznego. Zatem $X_1\sim \mathcal{H}(n_1,M_1,N_1)$, $X_2\sim \mathcal{H}(n_2,M_2,N_2)$ oraz proporcje są równe $p_1=M_1/N_1$, $p_2=M_2/N_2$. Znane parametry to rozmiary próbek $n_1$ i~$n_2$ oraz wielkości populacji $N_1$ i~$N_2$.

W celu wyprowadzenia wariancji rozkładu łącznego $X_1/n_1-X_2/n_2$ pod warunkiem $p_1=p_2$, zapiszmy łączną wariancję rozważanej zmiennej losowej, korzystając z~własności wariancji oraz tego, że $Cov(X_1,X_2)=0$ z~niezależności $X_1$ i~$X_2$
\begin{equation}
Var(X_1/n_1-X_2/n_2)=Var(X_1/n_1) + Var(X_2/n_2) = Var(X_1)/n_1^2+Var(X_2)/n_2^2.
\end{equation}
Wiemy, że wariancje $X_1$ i~$X_2$ są równe
\begin{align}
Var(X_1)=n_1 p_1 (1-p_1)(N_1-n_1)/(N_1-1),\\
Var(X_2)=n_2 p_2 (1-p_2)(N_2-n_2)/(N_2-1).
\end{align}
Pamiętając, że zakładamy równość $p_1=p_2$ zastąpmy oba parametry jednym równym $p$. Po podstawieniu otrzymujemy
\begin{equation}
\begin{split}
V_{X_1,X_2} = \frac{1}{n_1}p(1-p)\frac{N_1-n_1}{N_1-1} + \frac{1}{n_2}p(1-p)\frac{N_2-n_2}{N_2-1}&= \\
=p(1-p)&\left(\frac{N_1-n_1}{n_1(N_1-1)}+\frac{N_2-n_2}{n_2(N_2-1)}\right),
\end{split}
\end{equation}
przy czym $p$ to proporcja liczby osobników z~daną cechą do całości populacji w~rozkładzie łącznym.  Zatem jest ona równa $p=(X_1+X_2)/(n_1+n_2)$. Ostatecznie otrzymujemy
\begin{equation}
V_{X_1,X_2} = \left(\frac{N_1-n_1}{n_1(N_1-1)}+\frac{N_2-n_2}{n_2(N_2-1)}\right)\left(\frac{X_1+X_2}{n_1+n_2}\right)\left(1-\frac{X_1+X_2}{n_1+n_2}\right).
\end{equation}

\subsection{Test Z}
Test~Z jest oparty na centralnym twierdzeniu granicznym, które mówi, że pod warunkiem prawdziwości $H_0$ rozważana statystyka $Z_{X_1,X_2}$ jest w~przybliżeniu z~rozkładu standardowego normalnego $\mathcal{N}(0,1)$. Wtedy $p$-wartość wyraża się wzorem
\begin{equation}
P(|Z_{X_1,X_2}|\geq|Z_{k_1,k_2}|\ |H_0) = 2(1-\Phi(|Z_{k_1,k_2}|)),
\end{equation}
gdzie $\Phi$ oznacza dystrybuantę rozkładu $N(0,1)$. Test Z~odrzuca hipotezę zerową, gdy $p$-wartość jest mniejsza od poziomu istotności $\alpha$.

\subsection{Test E}
Test E opiera się o~rzeczywistą $p$-wartość, która, według artykułu K. Krishnamoorthy i~J. Thomson z~2002 roku, jest równa \cite{K.Krishnamoorthy2002}
\begin{equation}
\label{realpvalue}
\begin{split}
P(|Z_{X_1,X_2}|&\geq|Z_{k_1,k_2}|\ |H_0) = E_{X_1,X_2}(\1{|Z_{X_1,X_2}|\geq|Z_{k_1,k_2}|}\ |H_0) = \\
&= \sum_{x_1=L_1}^{U_1}\sum_{x_2=L_2}^{U_2} h(x_1;n_1,N_1p,N_1)h(x_2;n_2,N_2p,N_2)\1{|Z_{X_1,X_2}|\geq|Z_{k_1,k_2}|},
\end{split}
\end{equation}
gdzie $E_{X_1,X_2}$ to wartość oczekiwana łącznego rozkładu $(X_1,X_2)$, a~$p$ jest nieznaną wspólną proporcją pod warunkiem $H_0$. Nie jest możliwe policzenie $p$-wartości wprost ze wzoru (\ref{realpvalue}), ponieważ nie znamy parametru proporcji $p$. Krishnamoorthy i~Thomson (2002) zaproponowali estymator $p$-wartości \cite{K.Krishnamoorthy2002}
\begin{equation}
\label{estpvalue}
\begin{split}
P(|Z_{X_1,X_2}|\geq|Z_{k_1,k_2}|\ |H_0) =& \\ =\sum_{x_1=L_{x_1}}^{U_{x_1}}\sum_{x_2=L_{x_2}}^{U_{x_2}} h(x_1;&n_1,\hat{M_1},N_1)h(x_2;n_2,\hat{M_2},N_2)\1{|Z_{X_1,X_2}|\geq|Z_{k_1,k_2}|},
\end{split}
\end{equation}
przy czym $\hat{p}=(k_1+k_2)/(n_1+n_2)$, $\hat{M_i}=[N_i\hat{p}]$ oraz  $L_{x_i}=\max\{0,\hat{M_i}-N_i+n_i\}$, $U_{x_i}=\min\{n_i,\hat{M_i}\}$, $i=1,2$. Test odrzuca $H_0$ wtedy, gdy $p$-wartość wyliczona wg wzoru (\ref{estpvalue}) jest mniejsza od poziomu istotności $\alpha$.

\subsection{Moc testu}
Moc testu to prawdopodobieństwo odrzucenia hipotezy zerowej, gdy jest ona nieprawdziwa. Toteż jest ona wyznacznikiem dobrego testu. Większa wartość mocy oznacza lepszy test.

Moc obu testów można wyliczyć, korzystając z~funkcji prawdopodobieństwa rozkładu hipergeometrycznego. Dla testu~Z pod warunkiem hipotezy alternatywnej $H_1$ moc, zgodnie z~rozważaniami Krishnamoorthy i~Thomson (2002), jest równa \cite{K.Krishnamoorthy2002}
\begin{equation}
\label{powerZ}
\sum_{k_1=L_1}^{U_1}\sum_{k_2=L_2}^{U_2} h(k_1;n_1,M_1,N_1)h(k_2;n_2,M_2,N_2)\1{|Z_{k_1,k_2}|>z_{1-\alpha/2}},
\end{equation}
gdzie $L_i=\max\{0,M_i-N_i+n_i\}$ i~$U_i=\min\{n_i,M_i\}$, a~$z_{1-\alpha/2}$ oznacza kwantyl rozkładu normalnego standardowego rzędu $1-\alpha/2$.

Tymczasem dla testu E, wg Krishnamoorthy i~Thomson (2002), moc zdefiniowana jest następująco \cite{K.Krishnamoorthy2002}
\begin{equation}
\begin{split}
&\sum_{k_1=L_1}^{U_1}\sum_{k_2=L_2}^{U_2} h(k_1;n_1,M_1,N_1)h(k_2;n_2,M_2,N_2) \times \\
&\times \1{\sum_{x_1=L_{x_1}}^{U_{x_1}}\sum_{x_2=L_{x_2}}^{U_{x_2}} h(x_1;n_1,\hat{M_1},N_1)h(x_2;n_2,\hat{M_2},N_2) \1{|Z_{X_1,X_2}|\geq|Z_{k_1,k_2}|}\leq\alpha },
\end{split}
\end{equation}
gdzie wszelkie parametry oznaczają są takie same jak we wzorach (\ref{estpvalue}) i~(\ref{powerZ}).

\section{Test bez skończonej poprawki}
Dla testu bez poprawki na skończony rozmiar populacji, zamiast rozkładu hipergeometrycznego używamy dwumianowego, a~więc $X_1$ i~$X_2$ są niezależnymi zmiennymi losowymi o~rozkładzie Bernoullego $X_1\sim \mathcal{B}(n_1,p_1)$, $X_2\sim \mathcal{B}(n_2,p_2)$. Znane parametry to rozmiary próbek $n_1$ i~$n_2$.

Wariancję rozkładu łącznego $X_1/n_1-X_2/n_2$ pod warunkiem $p_1=p_2$ możemy wyprowadzić analogicznie jak w~sekcji \ref{r2:skonczonapoprawka}, wychodząc od łącznej wariancji rozważanej zmiennej losowej
\begin{equation}
Var(X_1/n_1-X_2/n_2) = Var(X_1)/n_1^2+Var(X_2)/n_2^2.
\end{equation}
Wariancje $X_1$ i~$X_2$ są równe
\begin{align}
Var(X_1)=n_1 p_1 (1-p_1),\\
Var(X_2)=n_2 p_2 (1-p_2).
\end{align}
Zastępując $p_1$ i~$p_2$ jednym parametrem równym $p$, otrzymujemy
\begin{equation}
V_{X_1,X_2} = p(1-p)/n_1 + p(1-p)/n_2 = p(1-p)(1/n_1+1/n_2)
\end{equation}
przy czym $p=(X_1+X_2)/(n_1+n_2)$.

Ten test jest, podobnie jak omówiony wcześniej test E, oparty o~estymator $p$-wartości, który, zgodnie z~artykułem Storer i~Kim z~1990 roku, jest równy \cite{Storer1990}
\begin{equation}
P(|Z_{X_1,X_2}|\geq|Z_{k_1,k_2}|\ |H_0) = \sum_{x_1=0}^{n_1}\sum_{x_2=0}^{n_2} b(x_1;n_1,\hat{p_1})b(x_2;n_2,\hat{p_2})\1{|Z_{X_1,X_2}|\geq|Z_{k_1,k_2}|},
\end{equation}
gdzie $\hat{p}=(k_1+k_2)/(n_1+n_2)$.
Test odrzuca hipotezę zerową, gdy $p$-wartość jest mniejsza od poziomu istotności $\alpha$.

Moc testu jest również podobna do mocy testu E. W~tym przypadku wyraża się wzorem
\begin{equation}
\begin{split}
\sum_{k_1=0}^{n}\sum_{k_2=0}^{n} b(k_1;n_1,&p_1)b(k_2;n_2,p_2) \times \\
\times &\1{\sum_{x_1=0}^{n_1} \sum_{x_2=0}^{n_2} b(x_1;n_1,\hat{p_1}) b(x_2;n_2,\hat{p_2}) \1{|Z_{X_1,X_2}|\geq|Z_{k_1,k_2}|} \leq\alpha}.
\end{split}
\end{equation}


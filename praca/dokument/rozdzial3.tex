\chapter{Analiza testów}
\label{r3}

W celu porównania testów opisanych w~rozdziale \ref{r2}, napisano programy, które wyliczają prawdopodobieństwo błędu I~rodzaju i~moc testu oraz generują wykresy dla różnych parametrów.

\section{Porównanie testów ze skończoną poprawką}
\label{r3:skonczonetesty}
Na początku zostały porównane testy ze skończoną poprawką. Rysunek \ref{sizeZE_n} przedstawia funkcję prawdopodobieństwa błędu I~rodzaju w~zależności od rozmiaru próbki dla różnych proporcji $p$. Po lewej stronie obserwacje z~obu populacji są tej samej wielkości $n_1=n_2=n$, natomiast po prawej różnią się, rozmiar pierwszej próbki ma ustaloną wartość. Pierwszym spostrzeżeniem jest to, że funkcja dla testu~E w~większości przypadków jest mniejsza niż dla testu Z. Dodatkowo dla testu~E prawdopodobieństwo błędu przekracza poziom istotności jedynie kilka razy i~to bardzo nieznacznie. Tymczasem dla testu~Z funkcja częściej przyjmuje wartości powyżej $\alpha$, dochodząc nawet do $0.95$, jak na wykresie \ref{sizeZE_p_0_05}. Warto wspomnieć także, że dla większych $p$ (0.1, 0.3) oraz wraz ze wzrostem wielkości próbki funkcje są bliżej siebie i~bardziej skupiają się przy poziomie istotności $\alpha$.

\begin{figure}[!h]
	\begin{subdiagrams}{sizeZE_p_0_05}{sizeZE_p_0_05_n1_5}
	\end{subdiagrams}
	
	\begin{subdiagrams}{sizeZE_p_0_1}{sizeZE_p_0_1_n1_10}
	\end{subdiagrams}
	
	\begin{subdiagrams}{sizeZE_p_0_3}{sizeZE_p_0_3_n1_5}
	\end{subdiagrams}
	\caption{Prawdopodobieństwo błędu I~rodzaju testów~Z i~E jako funkcja rozmiaru próbki $n$, przy zadanym poziomie istotności $\alpha=0.05$; $N_1=N_2=100$}
	\label{sizeZE_n}
\end{figure}

Rysunek \ref{sizeZE_p} obrazuje również prawdopodobieństwo błędu I~rodzaju, ale w~zależności od proporcji $p$. Analizując wykresy funkcji, można dojść do analogicznych wniosków, jak dla rysunku \ref{sizeZE_n}. W~każdym przypadku prawdopodobieństwo błędu I~rodzaju testu~Z jest większe od prawdopodobieństwa błędu testu E. Oprócz tego funkcja dla testu~E przekracza poziom istotności jedynie na wykresie \ref{sizeZE_n1_30_n2_15}, podczas gdy funkcja błędu testu~Z tylko w~jednym przypadku (\ref{sizeZE_n1_10_n2_6}) pozostaje całkowicie poniżej $\alpha$.

Podsumowując, dla małych rozmiarów próbek test~Z nie jest na poziomie istotności $\alpha$, co wynika z~zastosowania centralnego twierdzenia granicznego w~wyliczaniu $p$-wartości. Przybliżenie rozkładem normalnym działa dobrze tylko dla dużych prób. Wobec czego test~Z zbyt często odrzuca hipotezę zerową, gdy jest ona prawdziwa. 

\begin{figure}[!h]
	\begin{subdiagrams}{sizeZE_n_10}{sizeZE_n1_10_n2_6}
	\end{subdiagrams}
	
	\begin{subdiagrams}{sizeZE_n_20}{sizeZE_n1_20_n2_5}
	\end{subdiagrams}
	
	\begin{subdiagrams}{sizeZE_n_30}{sizeZE_n1_30_n2_15}
	\end{subdiagrams}
	\caption{Prawdopodobieństwo błędu I~rodzaju testów~Z i~E jako funkcja proporcji $p=M_1/N_1=M_2/N_2$, przy zadanym poziomie istotności $\alpha=0.05$; $N_1=N_2=100$}
	\label{sizeZE_p}
\end{figure}

Rysunek \ref{powerZE_n} zawiera wykresy mocy obu testów w~zależności od rozmiaru próbki dla różnych proporcji oraz rozmiarów populacji. Druga populacja ma ustaloną proporcję $p_2$, a~pierwsza przyjmuje trzy wartości różne od $p_2$. Dla obu testów zauważalny jest wzrost mocy wraz ze zwiększaniem się rozmiaru próbki. Ponadto moc testu wzrasta także dla proporcji bardziej oddalonych od siebie. W~każdym przypadku, dla próby $n=50$ i~różnicy między proporcjami $0.3$, moce mają wartość bliską $1$. 

Porównując, moc testu~Z jest większa od mocy testu E, jednakże różnice nie są znaczne. Ponadto, z~analizy błędu I~rodzaju, wiemy, że test~Z zbyt często odrzuca $H_0$, co wpływa na wyższą moc testu. Natomiast test E utrzymuje poziom istotności $\alpha$, wobec czego niewiele mniejsza moc testu nie oznacza, że test E jest gorszy. Wprost przeciwnie, patrząc na wartości prawdopodobieństwa błędu I~rodzaju, test E jest poprawny, podczas gdy test~Z nie spełnia swojej roli.

\begin{figure}[!h]
	\begin{subdiagrams}{powerZE_N1_30_N2_50_p_0_6}{powerZE_N_100_p_0_1}
	\end{subdiagrams}
	
	\begin{subdiagrams}{powerZE_N1_100_N2_200_p_0_1}{powerZE_N1_3000_N2_100_p_0_6}
	\end{subdiagrams}
	
	\caption{Moc testów~Z i~E jako funkcja rozmiaru próbki $n$}
	\label{powerZE_n}
\end{figure}

\section{Porównanie testu bez skończonej poprawki~Zb z~testem E}

W niniejszym podrozdziale porównano test~Zb z~testem E. Nie włączono do analizy testu Z, z~powodu wniosków opisanych w~podrozdziale \ref{r3:skonczonetesty}. Niniejsze rozważania zostały przeprowadzone, aby potwierdzić zasadność stosowania testu ze skończoną poprawką dla małych populacji lub o~stosunkowo dużej próbie.

Na rysunku \ref{sizeZbE_n} znajdują się wykresy prawdopodobieństwa błędu I~rodzaju analizowanych testów. Test E, jak stwierdzono w~podrozdziale \ref{r3:skonczonetesty}, nie przekracza poziomu istotności $\alpha$. Jednakże dla testu~Zb prawdopodobieństwo błędu I~rodzaju znacząco przekracza poziom istotności. Funkcja błędu~Zb przyjmuje wartości bliskie $0.2$, gdzie największa znajduje się na wykresie \ref{sizeZbE_p_0_1_n1_10}. Warto wspomnieć także, że funkcja jest największa w~przypadku małych próbek. Na wykresach \ref{sizeZbE_p_0_3} i~\ref{sizeZbE_p_0_3_n1_5} widać, że mniej więcej do $n=20$ test~Zb jest powyżej poziomu $\alpha$, następnie oscyluje, podobnie jak test E, w~okolicach poziomu istotności. 

Podsumowując, test~Zb nie jest adekwatny do rozważanego przypadku gdy $n$ jest małe. Odrzuca hipotezę zerową dużo częściej, niż powinien test na poziomie $\alpha$. Jest to spowodowane użyciem rozkładu dwumianowego w~formułowaniu statystyki testowej. Faktycznie dla dużych próbek różnica jest nieznaczna, natomiast w~sytuacji małej próbki rozbieżności są bardzo widoczne.

\begin{figure}[!h]
	\begin{subdiagrams}{sizeZbE_p_0_05}{sizeZbE_p_0_05_n1_5}
	\end{subdiagrams}
	
	\begin{subdiagrams}{sizeZbE_p_0_1}{sizeZbE_p_0_1_n1_10}
	\end{subdiagrams}
	
	\begin{subdiagrams}{sizeZbE_p_0_3}{sizeZbE_p_0_3_n1_5}
	\end{subdiagrams}
	\caption{Prawdopodobieństwo błędu I~rodzaju testów~Zb i~E jako funkcja rozmiaru próbki $n$, przy zadanym poziomie istotności $\alpha=0.05$; $N_1=N_2=100$}
	\label{sizeZbE_n}
\end{figure}

Rysunek \ref{powerZbE_n} przedstawia wykresy mocy rozważanych testów w~zależności od rozmiaru próbki dla różnych rozmiarów populacji oraz proporcji $p$. Moc testu~Zb jest niewiele większa od mocy testu E. Jednakże, podobnie jak w~przypadku testu Z, wynika to ze zbyt częstego odrzucania $H_0$. Biorąc pod uwagę ten fakt wyższa wartość testu~Zb nie świadczy o~jego poprawności.

\begin{figure}[!h]
	\begin{subdiagrams}{powerZbE_N1_30_N2_50_p_0_6}{powerZbE_N_100_p_0_1}
	\end{subdiagrams}
	
	\begin{subdiagrams}{powerZbE_N1_100_N2_200_p_0_1}{powerZbE_N1_3000_N2_100_p_0_6}
	\end{subdiagrams}
	
	\caption{Moc testów~Zb i~E jako funkcja rozmiaru próbki $n$}
	\label{powerZbE_n}
\end{figure}

Ostatecznie pokazano, że test bez skończonej poprawki nie powinien być stosowany w~przypadku małej populacji, ponieważ wtedy znacznie  przekracza poziom istotności $\alpha$. Najlepszym testem ze skończoną poprawką okazał się test E, który dla różnych przypadków utrzymuje poziom $\alpha$. 
\chapter{Analiza testów}

Do porównania testów napisałam programy, które wyliczają prawdopodobieństwo błędu I rodzaju oraz moc testu dla różnych parametrów. 



\begin{figure}[!h]
	\begin{subdiagrams}{sizeZE_p_0_05.png}{sizeZE_p_0_05_n1_5.png}
	\end{subdiagrams}
	
	\begin{subdiagrams}{sizeZE_p_0_1.png}{sizeZE_p_0_1_n1_10.png}
	\end{subdiagrams}
	
	\begin{subdiagrams}{sizeZE_p_0_3.png}{sizeZE_p_0_3_n1_5.png}
	\end{subdiagrams}
	\caption{P-stwo błędu I rodzaju jako funkcja rozmiaru próbki $n$, przy zadanym poziomie istotności $\alpha=0.05$; $N_1=N_2=100$}
	\label{sizeZE_n}
\end{figure}


\begin{figure}[!h]
	\begin{subdiagrams}{sizeZE_n_10.png}{sizeZE_n1_10_n2_6.png}
	\end{subdiagrams}
	
	\begin{subdiagrams}{sizeZE_n_20.png}{sizeZE_n1_20_n2_5.png}
	\end{subdiagrams}
	
	\begin{subdiagrams}{sizeZE_n_30.png}{sizeZE_n1_30_n2_15.png}
	\end{subdiagrams}
	\caption{P-stwo błędu I rodzaju jako funkcja parametru $p=M_1/N_1=M_2/N_2$, przy zadanym poziomie istotności $\alpha=0.05$; $N_1=N_2=100$}
	\label{sizeZE_p}
\end{figure}

\begin{figure}[!h]
	\begin{subdiagrams}{powerZE_N1_30_N2_50_p_0_6.png}{powerZE_N_100_p_0_1.png}
	\end{subdiagrams}
	
	\begin{subdiagrams}{powerZE_N1_100_N2_200_p_0_1.png}{powerZE_N1_3000_N2_100_p_0_6.png}
	\end{subdiagrams}
	
	\caption{Moc testu jako funkcja rozmiaru próbki $n$, przy zadanym poziomie istotności $\alpha=0.05$}
	\label{powerZE_n}
\end{figure}
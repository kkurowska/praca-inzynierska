\chapter{Schemat pobierania obserwacji}

W mojej pracy zajmuję się testami, które sprawdzają czy dwie populacje mają te same proporcje jakiejś badanej cechy. Zatem, aby wykonać taki test potrzebne są próbki z~obu populacji. Wartość obserwacji to ilość osobników z~badaną cechą w~próbce, toteż proporcja jest liczona jako stosunek wartości do wielkości próbki. Zakładamy również, że populacje są od siebie niezależne, tym samym próbki pochodzące z~tych populacji są także nie zależą od siebie. 


\section{Nieskończona populacja}
Gdy populacja jest bardzo duża, możemy traktować ją jako nieskończoną. Wobec tego losowanie kolejnych elementów próbki jest niezależne, czyli jest to losowanie ze zwracaniem. Zakładamy, że pobierane obserwacje pochodzą z~rozkładu Bernoullego $\mathcal{B}(n,p)$, gdzie $n$~to rozmiar próby z~nieskończonej populacji, a~$p$~to proporcja zdarzeń sprzyjających w~populacji. Funkcja prawdopodobieństwa zmiennej losowej $X$ z~rozkładu dwumianowego jest równa
\begin{equation}
b(k;n,p) = P(X=k) = \binom{n}{k} p^k (1-p)^{n-k},\ 0\leq k\leq n.
\end{equation}

\section{Skończona populacja}
\label{r1:skonczonapopulacja}
Kiedy populację rozważamy jako skończoną kolejne elementy próbki są losowane bez zwracania. Oznacza to, że prawdopodobieństwo sukcesu zmienia się w~trakcie pobierania elementów obserwacji w~zależności od poprzednich. W~takiej sytuacji próbka pochodzi z~rozkładu hipergeometrycznego $\mathcal{H}(n,M,N)$. Przy czym $n$~jest rozmiarem próbki, $M$~ilością osobników w~populacji z~daną cechą, a~$N$~rozmiarem populacji. Zmienna losowa $X$~z~rozkładu hipergeometrycznego określa ma funkcja prawdopodobieństwa określoną wzorem
\begin{equation}
h(k;n,M,N) = P(X=k) = \frac{\binom{M}{k} \binom{N-M}{n-k}}{\binom{N}{n}},\ L\leq k\leq U,
\end{equation}
gdzie $L=\max\{0,M-N+n\}$ i~$U=\min\{n,M\}$.

Zauważmy, że wzór funkcji $h$ jest dość intuicyjny. Klasyczna definicja prawdopodobieństwa określa szansę zajścia zdarzenia~$A$ jako iloraz liczby zdarzeń elementarnych w~$A$ przez liczbę zdarzeń elementarnych w~$\Omega$, czyli $P(A) = |A|/|\Omega|$. W~tym przypadku zdarzeniem~$A$ jest to, że w~próbce będzie $k$~osobników z~daną cechą. Zatem ilość zdarzeń elementarnych w~$A$ to kombinacje. Na jak wiele różnych sposobów możemy wybrać $k$~osobników z~$M$~wszystkich posiadających daną cechę $\binom{M}{k}$ razy możliwość wyborów pozostałych osobników z~reszty populacji $\binom{N-M}{n-k}$. Natomiast zbiór wszystkich zdarzeń elementarnych w~$\Omega$ to po prostu wybór losowej próbki $n$~osobników z~$N$-elementowej populacji $\binom{N}{n}$. Również ograniczenia nałożone na~$k$ są naturalne. Dolne ograniczenie~$L$ jest równe maksimum z~$0$ i~$M-N+n$. Zatem będzie ono niezerowe, gdy $M-N+n>0$. Przekształcając nierówność, otrzymujemy $n>N-M$. Taka postać jasno pokazuje, że jest to przypadek, w~którym wielkość próbki przekracza ilość osobników w~populacji bez badanej cechy. W~konsekwencji czego mamy pewność, że w~próbce będzie przynajmniej tyle osobników z~daną cechą, ile wynosi różnica $n-(N-M)$. Ograniczenie górne jest mniej skomplikowane. Jest ono równe minimum z~$n$~i~$M$, co jest oczywiste, ponieważ nie może być więcej osób w~próbce z~daną cechą niż w~całej populacji. Cała ta analiza pokazuje, że rozkład hipergeometryczny jest ściśle związany z~rozmiarem populacji.

\section{Porównanie rozkładów}

Rozkład hipergeometryczny daje bardzo podobne wyniki do rozkładu Bernoullego, gdy populacja jest duża, a~próbka stosunkowo mała. Jednakże w~przeciwnym wypadku widać znaczne różnice między tymi rozkładami. W~szczególności obserwujemy, że zmienna losowa z~rozkładu dwumianowego może osiągać wartości od $0$ do $n$, ponieważ podczas brania próbki zakładamy losowanie ze zwracaniem. Natomiast w~rozkładzie hipergeometrycznym kolejne losowania są od siebie zależne, prawdopodobieństwo wylosowania zmienia się w~zależności od tego, co już wcześniej znalazło się w~próbce.

Rozważmy to na medycznym przykładzie. Załóżmy, że jest na świecie $20$ osób, które są chore na jakąś bardzo rzadką chorobę oraz że $25\%$ z~nich ma szanse na wyzdrowienie. Chcemy dowiedzieć się, ile osób spośród przebadanych może wyzdrowieć. Weźmy $3$~różne próbki o~wielkościach~$n$ równych odpowiednio $10$, $17$, $20$. Możemy tę sytuację opisać za pomocą rozkładem Bernoullego, wtedy badana zmienna losowa będzie z~rozkładu $\mathcal{B}(n,0.25)$. Drugim sposobem jest rozkład hipergeometryczny, wtedy zmienna losowa jest z~rozkładu $\mathcal{H}(n,4,20)$. Na rysunkach \ref{pmf10}\ppauza \ref{pmf20} jest zobrazowana funkcja prawdopodobieństwa dla wymienionych przypadków.

\begin{diagrams}{b_pmf10.png}{hg_pmf10.png}
	\caption{Funkcja prawdopodobieństwa dla $n=10$}
	\label{pmf10}
\end{diagrams}

\begin{diagrams}{b_pmf17.png}{hg_pmf17.png}
	\caption{Funkcja prawdopodobieństwa dla $n=17$}
	\label{pmf17}
\end{diagrams}

\begin{diagrams}{b_pmf20.png}{hg_pmf20.png}
	\caption{Funkcja prawdopodobieństwa dla $n=20$}
	\label{pmf20}
\end{diagrams}

Im większa próbka tym widać większą różnicę między funkcjami obu rozkładów. Funkcja dla rozkładu hipergeometrycznego nie ma innych argumentów niż te, które są możliwe, natomiast funkcja dla rozkładu dwumianowego jest liczona również dla nieprawdopodobnych argumentów, ponadto wtedy jej wartość jest niezerowa. Bardzo skrajny przypadek przedstawia rysunek \ref{pmf20}, próbka równa się populacji, czyli tak naprawdę wiemy już wszystko. Wykres \ref{hg_pmf20.png} idealnie obrazuje nam ten przypadek. Pewne jest to, że w~$20$ osobach będzie dokładnie $5$, które mogą wyzdrowieć. Natomiast wykres \ref{b_pmf20.png} kompletnie nie pokazuje tego. Głównie ze względu na już wspomniane uwzględnianie nieprawdopodobnych argumentów, przez co prawdopodobieństwo rozkłada się na pozostałe przypadki. Warto również zaznaczyć, że im większa próbka, tym prawdopodobieństwo, że $X=5$ dla rozkładu Bernoullego jest coraz mniejsze, ponieważ mamy więcej elementów w~obserwacji. Tymczasem dla rozkładu hipergeometrycznego jest wręcz odwrotnie, to prawdopodobieństwo rośnie, aż w~końcu osiąga wartość $1$. Co jest dużo bardziej logiczne, bo im więcej przebadaliśmy osobników, tym więcej wiemy o~próbce i~jest bardziej prawdopodobne, że znajdzie się w~niej aż $5$ szczęśliwych pacjentów.

Spójrzmy jeszcze, jak wyglądaja średnia i wariancja dla rozważanych rozkładów. \newline
Gdy $X\sim\mathcal{B}(n,p)$
\begin{equation}
E(X)=np,\quad Var(X)=np(1-p).
\end{equation}
Podczas gdy $X\sim\mathcal{H}(n,M,N)$
\begin{equation}
E(X)=n\frac{M}{N},\quad Var(X)=n\frac{M}{N}\left(1-\frac{M}{N}\right)\frac{N-n}{N-1}.
\end{equation}
Parametr $p$ rozkładu dwumianowego odpowiada ilorazowi parametrów $M/N$ w~rozkładzie hipergeometrycznym. Wobec tego średnie obu rozkładów są sobie równe, a wariancje różni dodatkowy składnik w~rozkładzie hipergeometrycznym $(N-n)/(N-1)$. Przeanalizujmy jak ten czynnik wpływa na zróżnicowanie rozkładów. Rysunek \ref{var} przedstawia wykresy wariancji dla rozważanego przykładu w~zależności od wielkości obserwacji. W~przypadku rozkładu dwumianowego wariancja stale rośnie wraz ze wzrostem próbki. Ostatecznie, gdy już przebadamy wszystkich chorych pacjentów, ma ona największą wartość. Myśląc zdroworozsądkowo, nie jest to poprawny wynik. Oczekujemy raczej, że gdy przebadamy już wszystkich, wariancja osiągnie wartość $0$, ponieważ nie ma wtedy losowości. Taki rezultat daje nam wykres \ref{hg_var.png}. Funkcja na początku rośnie, ale gdy wielkość próbki przekroczy połowę rozmiaru populacji, wariancja zaczyna maleć aż do zera. Odzwierciedla to fakt, że gdy coraz więcej wiemy o~populacji, losowość uzyskanych wyników maleje.

\begin{diagrams}{b_var.png}{hg_var.png}
	\caption{Wariancja w~zależności od rozmiaru próbki}
	\label{var}
\end{diagrams}
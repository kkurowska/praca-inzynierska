\chapter{Przedstawienie testów}

W tym rozdziale chciałabym opisać dwa testy wykorzystujące rozkład hipergeometryczny oraz test oparty na rozkładzie dwumianowym. Omówię także sposób liczenia prawdopodobieństwa błędu I rodzaju i mocy testu.

\section{Sformułowanie problemu}
Załóżmy, że $X_1$ i $X_2$ są niezależnymi zmiennymi losowymi o rozkładzie hipergeometrycznym $X_1\sim h(n_1,M_1,N_1)$, $X_2\sim h(n_2,M_2,N_2)$. Zaobserwowane wartości $X_1$ i $X_2$ oznaczmy odpowiednio $k_1$ i $k_2$ oraz proporcje $p_1=M_1/N_1$, $p_2=M_2/N_2$. Będę zajmować się testowaniem hipotez
\begin{equation}
H_0{:}\ p_1=p_2\quad \text{przeciwko} \quad H_1{:}\ p_1\neq p_2,
\end{equation}
w oparciu o $(k_1,n_1,N_1)$ i $(k_2,n_2,N_2)$.
Rozważmy unormowaną statystykę
\begin{equation}
Z_{X_1,X_2} = \frac{X_1/n_1-X_2/n_2}{\sqrt{V_{X_1,X_2}}},
\end{equation}
gdzie estymator wariancji pod warunkiem zachodzenia $H_0$ jest równy
\begin{equation}
V_{X_1,X_2} = \left(\frac{N_1-n_1}{n_1(N_1-1)}+\frac{N_2-n_2}{n_2(N_2-1)}\right)\left(\frac{X_1+X_2}{n_1+n_2}\right)\left(1-\frac{X_1+X_2}{n_1+n_2}\right).
\end{equation}
Wartość statystyki dla $k_1$ i $k_2$ będę oznaczać jako $Z_{k_1,k_2}$. Jest ona wyliczana według powyższych wzorów, zamieniając $X_1$ i $X_2$ wartościami obserwacji.

\section{Test Z}
Ten test jest oparty na centralnym twierdzeniu granicznym, które mówi, że pod warunkiem $H_0$ w przybliżeniu rozważana statystyka $Z_{X_1,X_2}$ jest z rozkładu normalnego standaryzowanego $N(0,1)$. Wtedy $p$-wartość wyraża się wzorem
\begin{equation}
P(|Z_{X_1,X_2}|\geq|Z_{k_1,k_2}|\ |H_0) = 2(1-\Phi(|Z_{k_1,k_2}|)),
\end{equation}
gdzie $\Phi()$ oznacza dystrybuantę rozkładu $N(0,1)$. Test Z odrzuca hipotezę zerową, gdy $p$-wartość jest mniejsza od poziomu istotności $\alpha$.

\section{Test E}
W tym przypadku opieramy się o rzeczywistą $p$-wartość, która jest równa
\begin{equation}
\label{realpvalue}
\begin{split}
P(|Z_{X_1,X_2}|\geq|Z_{k_1,k_2}|\ |H_0) =& E_{X_1,X_2}(\1{|Z_{X_1,X_2}|\geq|Z_{k_1,k_2}|}\ |H_0) = \\
= \sum_{x_1=L_1}^{U_1}\sum_{x_2=L_2}^{U_2}& h(x_1;n_1,N_1p,N_1)h(x_2;n_2,N_2p,N_2)\1{|Z_{X_1,X_2}|\geq|Z_{k_1,k_2}|},
\end{split}
\end{equation}
gdzie $E_{X_1,X_2}$ to wartość oczekiwana łącznego rozkładu $(X_1,X_2)$, a $p$ jest nieznaną wspólną proporcją pod warunkiem $H_0$. Nie jest możliwe policzenie $p$-wartości wprost ze wzoru (\ref{realpvalue}), ponieważ nie znamy parametru proporcji $p$. W artykule \cite{K.Krishnamoorthy2002} zaproponowany jest estymator $p$-wartości
\begin{equation}
\label{estpvalue}
\begin{split}
P(|Z_{X_1,X_2}|\geq|Z_{k_1,k_2}|\ |H_0) =& \\ =\sum_{x_1=L_{x_1}}^{U_{x_1}}\sum_{x_2=L_{x_2}}^{U_{x_2}}& h(x_1;n_1,\hat{M_1},N_1)h(x_2;n_2,\hat{M_2},N_2)\1{|Z_{X_1,X_2}|\geq|Z_{k_1,k_2}|},
\end{split}
\end{equation}
przy czym $\hat{p}=(k_1+k_2)/(n_1+n_2)$, $\hat{M_i}=[N_i\hat{p}]$, $L_{x_i}=\max\{0,\hat{M_i}-N_i+n_i\}$, $U_{x_i}=\min\{n_i,\hat{M_i}\}$, $i=1,2$. Test odrzuca $H_0$ wtedy, gdy $p$-wartość wyliczona wg wzoru (\ref{estpvalue}) jest mniejsza od poziomu istotności $\alpha$.

\section{Błąd I rodzaju}
Błąd I rodzaju to odrzucenie hipotezy zerowej, gdy jest ona prawdziwa. Prawdopodobieństwo tego błędu można wyliczyć, losując próbki z populacji, gdy $p_1=p_2$ i sprawdzając, ile razy zostaną odrzucone. Dla dużej ilości próbek prawdopodobieństwo błędu I rodzaju powinno być w okolicy poziomu istotności testu.

\section{Moc testu}
Przypomnę, że moc testu to prawdopodobieństwo odrzucenia hipotezy zerowej, gdy jest ona nieprawdziwa. Toteż jest ona wyznacznikiem dobrego testu. Większa wartość mocy oznacza lepszy test.

Moc obu testów można wyliczyć, korzystając z funkcji prawdopodobieństwa rozkładu hipergeometrycznego. Dla testu Z pod warunkiem hipotezy alternatywnej $H_1$ moc jest równa
\begin{equation}
\label{powerZ}
\sum_{k_1=L_1}^{U_1}\sum_{k_2=L_2}^{U_2} h(k_1;n_1,M_1,N_1)h(k_2;n_2,M_2,N_2)\1{|Z_{k_1,k_2}|>z_{1-\alpha/2}},
\end{equation}
gdzie $L_i=\max\{0,M_i-N_i+n_i\}$ i $U_i=\min\{n_i,M_i\}$, a $z_{1-\alpha/2}$ oznacza kwantyl rozkładu normalnego standardowego rzędu $1-\alpha/2$.

Tymczasem dla testu E moc zdefiniowana jest następująco
\begin{equation}
\begin{split}
&\sum_{k_1=L_1}^{U_1}\sum_{k_2=L_2}^{U_2} h(k_1;n_1,M_1,N_1)h(k_2;n_2,M_2,N_2) \times \\
& \times \1{\sum_{x_1=L_{x_1}}^{U_{x_1}}\sum_{x_2=L_{x_2}}^{U_{x_2}} h(x_1;n_1,\hat{M_1},N_1)h(x_2;n_2,\hat{M_2},N_2) \1{|Z_{X_1,X_2}|\geq|Z_{k_1,k_2}|}\leq\alpha },
\end{split}
\end{equation}
gdzie wszelkie parametry oznaczają to samo co we wzorach (\ref{estpvalue}) i (\ref{powerZ}).

\section{Test bez skończonej poprawki}
W tym przypadku zamiast rozkładu hipergeometrycznego używamy dwumianowego, a więc $X_1$ i $X_2$ są niezależnymi zmiennymi losowymi o rozkładzie Bernoullego $X_1\sim B(n_1,p_1)$, $X_2\sim B(n_2,p_2)$. Unormowana statystyka w tym przypadku przyjmuje postać
\begin{equation}
Z_{X_1,X_2} = \frac{X_1/n_1-X_2/n_2}{\sqrt{V_{X_1,X_2}}},
\end{equation}
gdzie estymator wariancji pod warunkiem $H_0$ jest równy
\begin{equation}
V_{X_1,X_2} = \sqrt{p(1-p)(1/n_1+1/n_2)},
\end{equation}
gdzie $p=(X_1+X_2)/(n_1+n_2)$.
Wartość statystyki $Z_{k_1,k_2}$ jest wyliczana według powyższych wzorów, wstawiając obserwacje $k_1$ i $k_2$ w miejsca zmiennych losowych.

Ten test jest, podobnie jak omówiony wcześniej test Z, oparty na centralnym twierdzeniu granicznym. Czyli rozważana statystyka $Z_{X_1,X_2}$ ma rozkład standardowy normalny $N(0,1)$. Wtedy $p$-wartość wyraża się wzorem
\begin{equation}
P(|Z_{X_1,X_2}|\geq|Z_{k_1,k_2}|\ |H_0) = 2(1-\Phi(|Z_{k_1,k_2}|)),
\end{equation}
Test odrzuca hipotezę zerową, gdy $p$-wartość jest mniejsza od poziomu istotności $\alpha$.

Prawdopodobieństwo błędu I rodzaju jest liczone analogicznie jak w przypadku poprzednich testów. Aczkolwiek moc testu jest liczona inaczej, ze względu na inny rozkład zmiennych losowych. W tym przypadku będzie ona równa
\begin{equation}
\sum_{k_1=0}^{n}\sum_{k_2=0}^{n} b(k_1;n_1,p_1)b(k_2;n_2,p_2)\1{|Z_{k_1,k_2}|>z_{1-\alpha/2}},
\end{equation}
przy czym $b(k;n,p)$ oznacza funkcję prawdopodobieństwa rozkładu dwumianowego określoną wzorem
\begin{equation}
b(k;n,p) = P(X=k) = \binom{n}{k} p^k (1-p)^{n-k},\ 0\leq k\leq n.
\end{equation}

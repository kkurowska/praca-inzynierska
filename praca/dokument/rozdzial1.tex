\chapter{Schemat pobierania obserwacji}

W niniejszej pracy analizowane są testy, które sprawdzają, czy dwie populacje mają te same proporcje jakiejś badanej cechy. Zatem, aby wykonać test, potrzebne są próbki z~obu populacji. Proporcja jest liczona jako stosunek wartości obserwacji do wielkości próbki, gdzie Wartość obserwacji to ilość osobników z~badaną cechą w~próbce. Zakładamy również, że populacje są od siebie niezależne, tym samym próbki pochodzące z~tych populacji także nie zależą od siebie. 


\section{Nieskończona populacja}
Gdy populacja jest bardzo duża, możemy traktować ją jako nieskończoną. Wobec tego losowanie kolejnych elementów próbki jest niezależne, czyli jest to losowanie ze zwracaniem. Zakładamy, że pobierane obserwacje pochodzą z~rozkładu Bernoulliego $\mathcal{B}(n,p)$, gdzie $n$~to rozmiar próby z~nieskończonej populacji, a~$p$~to proporcja zdarzeń sprzyjających w~populacji. Funkcja prawdopodobieństwa zmiennej losowej $X$ z~rozkładu dwumianowego jest równa
\begin{equation}
b(k;n,p) = P(X=k) = \binom{n}{k} p^k (1-p)^{n-k},\ 0\leq k\leq n.
\end{equation}

\section{Skończona populacja}
\label{r1:skonczonapopulacja}
Kiedy populację rozważamy jako skończoną, kolejne elementy próbki są losowane bez zwracania. Oznacza to, że prawdopodobieństwo sukcesu zmienia się w~trakcie pobierania elementów obserwacji w~zależności od poprzednich. W~takiej sytuacji próbka pochodzi z~rozkładu hipergeometrycznego $\mathcal{H}(n,M,N)$. Przy czym $n$~jest rozmiarem próbki, $M$~ilością osobników w~populacji z~daną cechą, a~$N$~rozmiarem populacji. Zmienna losowa~$X$ z~rozkładu hipergeometrycznego ma funkcję prawdopodobieństwa określoną wzorem
\begin{equation}
\label{hg}
h(k;n,M,N) = P(X=k) = \frac{\binom{M}{k} \binom{N-M}{n-k}}{\binom{N}{n}},\ L\leq k\leq U,
\end{equation}
gdzie
\begin{equation}
\label{ograniczenia}
L=\max\{0,M-N+n\},\quad U=\min\{n,M\}.
\end{equation}

Zauważmy, że wzór (\ref{hg}) ma prostą interpretację. Klasyczna definicja prawdopodobieństwa określa szansę zajścia zdarzenia~$A$ jako iloraz liczby zdarzeń elementarnych w~$A$ przez liczbę zdarzeń elementarnych w~$\Omega$, czyli $P(A) = |A|/|\Omega|$. W~rozważanym przypadku zdarzeniu~$A$ odpowiada sytuacja, w~której próbka będzie zawierać $k$~osobników z~daną cechą. Zatem ilość zdarzeń elementarnych w~$A$ to iloczyn dwóch kombinacji. Wybór $k$~osobników z~$M$~posiadających daną cechę $\binom{M}{k}$ mnożymy przez możliwość wyborów pozostałych osobników z~reszty populacji $\binom{N-M}{n-k}$. Natomiast zbiór wszystkich zdarzeń elementarnych w~$\Omega$ to wybór losowej próbki $n$~osobników z~$N$-elementowej populacji $\binom{N}{n}$. Ograniczenia nałożone na~$k$ są również naturalne. Dolne ograniczenie~$L$ jest równe maksimum z~$0$ i~$M-N+n$. Będzie ono niezerowe, gdy $M-N+n>0$. Przekształcając nierówność, otrzymujemy $n>N-M$, czyli rozmiar próbki jest większy od ilości osobników w~populacji bez badanej cechy. W~konsekwencji czego mamy pewność, że w~próbce będzie przynajmniej tyle osobników z~daną cechą, ile wynosi różnica $n-(N-M)$. Ograniczenie górne jest równe minimum z~$n$~i~$M$, co wynika z~faktu, że nie może być więcej osób w~próbce z~daną cechą niż w~całej populacji. Analiza wzoru (\ref{hg}) pokazuje, że rozkład hipergeometryczny jest ściśle związany z~rozmiarem populacji.

\section{Porównanie rozkładów}

Rozkład hipergeometryczny daje bardzo podobne wyniki do rozkładu Bernoulliego, gdy populacja jest duża, albo~próbka stosunkowo mała. Natomiast przy małej populacji i~dużej próbce różnica między tymi rozkładami jest znaczna.

Rozważmy to na medycznym przykładzie. Załóżmy, że jest na świecie $20$ osób, które są chore na jakąś bardzo rzadką chorobę oraz że $25\%$ z~nich ma szanse na wyzdrowienie. Chcemy dowiedzieć się, ile osób spośród przebadanych może wyzdrowieć. Weźmy $3$~różne próbki o~wielkościach~$n$ równych odpowiednio $10$, $17$, $20$. Możemy tę sytuację opisać za pomocą rozkładu Bernoulliego, wtedy badana zmienna losowa jest z~rozkładu $\mathcal{B}(n,0.25)$. Drugim sposobem jest rozkład hipergeometryczny, wtedy zmienna losowa $X\sim\mathcal{H}(n,5,20)$. Rysunki \ref{pmf10}\ppauza\ref{pmf20} przedstawiają funkcję prawdopodobieństwa dla wymienionych przypadków.

Przeanalizujmy jakie wartości mogą przyjmować zmienne losowe z~obu rozkładów. W~rozkładzie dwumianowym zmienna w~każdym z~przypadków przyjmuje wartości od $0$ do $n$. Na przykład na wykresie \ref{b_pmf10} argumenty funkcji to zbiór $\{0,1,\ldots,9,10\}$. Zauważmy, że prawdopodobieństwo wyzdrowienia sześciu osób wynosi około $0.09$. W~przykładzie było powiedziane, że $25\%$ z~$20$ może wyzdrowieć, czyli maksymalnie $5$ osób. Wobec tego wartość $6$ nie jest prawdopodobna. Przedstawiona rozbieżność wynika z~założenia, że pobieranie próbki to losowanie ze zwracaniem, czyli w~próbce mogą znaleźć się dwie te same osoby. Natomiast w~rozkładzie hipergeometrycznym zmienną losową ograniczają $L$ i~$U$, zdefiniowane we wzorze (\ref{ograniczenia}), które uwzględniają wielkość próbki. Porównując, na wykresie \ref{hg_pmf10} zmienna losowa nie przyjmuje wartości większej niż $5$. Dodatkowo na wykresie \ref{hg_pmf17} najmniejszą wartością jest $2$, ponieważ populacja zawiera $15$ osób, które nie wyzdrowieją, zatem w~$17$-to osobowej próbce jest pewne, że przynajmniej dwie będą zdrowe. Kolejne losowania są od siebie zależne, więc nie możemy drugi raz wylosować tej samej osoby.

\begin{diagrams}{b_pmf10}{hg_pmf10}
	\caption{Funkcje prawdopodobieństwa $b(k;10,0.25)$ oraz $h(k;10,5,20)$}
	\label{pmf10}
\end{diagrams}

\begin{diagrams}{b_pmf17}{hg_pmf17}
	\caption{Funkcje prawdopodobieństwa $b(k;17,0.25)$ oraz $h(k;17,5,20)$}
	\label{pmf17}
\end{diagrams}

\begin{diagrams}{b_pmf20}{hg_pmf20}
	\caption{Funkcje prawdopodobieństwa $b(k;20,0.25)$ oraz $h(k;20,5,20)$}
	\label{pmf20}
\end{diagrams}

Różnica między funkcjami obu rozkładów rośnie wraz ze wzrostem próbki. Skrajny przypadek przedstawia rysunek \ref{pmf20}, na którym próbka równa się populacji. Wykres \ref{hg_pmf20} idealnie obrazuje przypadek $n=N$. Pewne jest to, że w~$20$ osobach będzie dokładnie $5$, które mogą wyzdrowieć. Natomiast na wykresie \ref{b_pmf20} prawdopodobieństwo, że $X=5$ wynosi jedynie $0.2$, ponieważ funkcja prawdopodobieństwa dla rozkładu dwumianowego nie uwzględnia wielkości populacji. Warto również zaznaczyć, że im większa próbka, tym $P(X=5)$ dla rozkładu Bernoulliego jest coraz mniejsze, ponieważ mamy więcej elementów w~obserwacji. Tymczasem dla rozkładu hipergeometrycznego jest odwrotnie, funkcja prawdopodobieństwa dla $k=5$ rośnie, aż w~końcu osiąga wartość $1$. Gdy przebadamy więcej osobników, wzrasta nasza wiedza o~próbce oraz jest bardziej prawdopodobne to, że znajdzie się w~niej aż $5$ zdrowych pacjentów.

Spójrzmy jeszcze, jak wyglądają średnia i~wariancja dla rozważanych rozkładów. \newline
Gdy $X\sim\mathcal{B}(n,p)$, to
\begin{equation}
E(X)=np,\quad Var(X)=np(1-p).
\end{equation}
Podczas gdy $X\sim\mathcal{H}(n,M,N)$, to
\begin{equation}
E(X)=n\frac{M}{N},\quad Var(X)=n\frac{M}{N}\left(1-\frac{M}{N}\right)\frac{N-n}{N-1}.
\end{equation}
Parametr $p$ rozkładu dwumianowego odpowiada ilorazowi parametrów $M/N$ w~rozkładzie hipergeometrycznym. Wobec tego średnie obu rozkładów są sobie równe, ale wariancje różni dodatkowy składnik w~rozkładzie hipergeometrycznym $(N-n)/(N-1)$. Przeanalizujmy, jak ten czynnik wpływa na zróżnicowanie rozkładów. Rysunek \ref{var} przedstawia wykresy wariancji dla rozważanego przykładu w~zależności od wielkości obserwacji. W~przypadku rozkładu dwumianowego wariancja stale rośnie wraz ze wzrostem próbki. Ostatecznie, gdy $n=20$ jest ona największa. Jednakże, gdy przebadamy całą populację, oczekiwanym rezultatem jest wartość zerowa wariancji, ponieważ nie ma wtedy losowości. Taki wynik daje nam wykres \ref{hg_var}. Funkcja na początku rośnie, ale gdy wielkość próbki przekroczy połowę rozmiaru populacji, wariancja zaczyna maleć aż do zera. Odzwierciedla to fakt, że gdy coraz więcej wiemy o~populacji, losowość uzyskanych wyników maleje.

\begin{diagrams}{b_var}{hg_var}
	\caption{Wariancja rozkładów Bernoulliego i~hipergeometrycznego w~zależności od rozmiaru próbki}
	\label{var}
\end{diagrams}
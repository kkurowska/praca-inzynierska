\documentclass[12pt,a4paper,twoside]{book}
\usepackage{classicthesis}
\usepackage[a4paper, top=2.5cm, bottom=2.5cm, left=2.5cm, right=2.5cm]{geometry}

\usepackage{polski}
\usepackage[utf8]{inputenc}
\usepackage[T1]{fontenc}
\usepackage{indentfirst}

\usepackage{wallpaper}
\ULCornerWallPaper{1}{background.pdf}

\usepackage{tabularx}
\usepackage{paralist}

\usepackage{graphicx}

\begin{document}
\newgeometry{top=4.8cm, left=5cm, bottom=2.5cm}
\begin{titlepage}
\noindent\textbf{\large Wydział Matematyki}
\par\medskip\noindent
Kierunek: Matematyka Stosowana
\par\noindent
Specjalność: \textit{nie dotyczy}
\vspace*{36pt}
\begin{center}
\LARGE Praca dyplomowa --- inżynierska
\end{center}
\vspace*{24pt}
\begin{center}
\uppercase{\Large\bfseries%
Testowanie hipotez i estymacja w~sytuacji populacji sko\'nczonego rozmiaru}
\end{center}
\vspace*{12pt}
\begin{center}
Kinga Kurowska
\end{center}
\vspace*{12pt}
\begin{flushright}
Słowa kluczowe:\par\noindent
testowanie hipotez\par\noindent
przedziały ufności\par\noindent
zastosowanie w medycynie\par\noindent
\end{flushright}
\begin{flushleft}
Krótkie streszczenie:\par
W~pracy został rozważony problem testowania hipotez w~sytuacji populacji skończonego rozmiaru. Przedstawiono i~porównano dwa schematy pobierania obserwacji, dla nieskończonej oraz skończonej populacji. Zostały opisane trzy testy statystyczne oraz przeprowadzone badania symulacyjne w~celu porównania testów. Ostatecznie został wybrany jeden test, który jest najlepszy dla małych populacji.
\smallskip
\end{flushleft}
\begin{tabularx}{\textwidth}{|l|c|X|X|}
\hline
{\footnotesize Opiekun pracy} & {\small dr inż.\ Andrzej Giniewicz} &  &  \\
\cline{2-4}
{\footnotesize dyplomowej} & \textit{\footnotesize Stopień naukowy, imię i nazwisko} & \textit{\footnotesize Ocena} & \textit{\footnotesize Podpis} \\
\hline
\end{tabularx}
\smallskip
\begin{flushleft}
\small\itshape
Do celów archiwalnych pracę dyplomową zakwalifikowano do: *
\begin{compactenum}[{\enskip\enskip} a)\enskip]
\item kategorii A (akta wieczyste),
\item kategorii BE 50 (po 50 latach podlegające ekspertyzie).
\end{compactenum}
* niepotrzebne skreślić
\end{flushleft}
\vspace*{12pt}
\begin{flushright}
\fbox{\footnotesize\enskip\textit{pieczątka wydziałowa}\enskip}
\end{flushright}
\vspace*{12pt}
\begin{center}
Wrocław, rok 2017
\end{center}
\end{titlepage}
\restoregeometry
\end{document}
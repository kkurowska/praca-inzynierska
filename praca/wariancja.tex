\documentclass[12pt]{mwrep}

\usepackage{polski}
\usepackage[utf8]{inputenc}
\usepackage{amsmath,amssymb,amsthm}


\begin{document}

Wyprowadzenie wariancji rozkładu łącznego $X_1/n_1-X_2/n_2$ pod warunkiem $p_1=p_2$, gdzie $X_1$ i $X_2$ to niezależne zmienne losowe z rozkładu hipergeometrycznego, a $p_1=M_1/N_1$, $p_2=M_2/N_2$.

Zapiszmy łączną wariancję rozważanej zmiennej losowej, korzystając z własności wariancji oraz tego, że $Cov(X_1,X_2)=0$ z niezależności $X_1$ i $X_2$
\begin{equation}
Var\left(\frac{X_1}{n_1}-\frac{X_2}{n_2}\right)=Var\left(\frac{X_1}{n_1}\right) + Var\left(\frac{X_2}{n_2}\right)=\frac{1}{n_1^2}Var(X_1)+\frac{1}{n_2^2}Var(X_2).
\end{equation}
Wiemy, że wariancje $X_1$ i $X_2$ są równe
\begin{align}
Var(X_1)=n_1 p_1 (1-p_1)(N_1-n_1)/(N_1-1),\\
Var(X_2)=n_2 p_2 (1-p_2)(N_2-n_2)/(N_2-1).
\end{align}
Pamiętając, że zakładamy równość $p_1=p_2$ zastąpmy oba parametry jednym $p$. Po podstawieniu otrzymujemy
\begin{equation}
\begin{split}
Var\left(\frac{X_1}{n_1}-\frac{X_2}{n_2}\right)=\frac{1}{n_1}p(1-p)\frac{N_1-n_1}{N_1-1} + \frac{1}{n_2}p(1-p)\frac{N_2-n_2}{N_2-1}= \\
=p(1-p)\left(\frac{N_1-n_1}{n_1(N_1-1)}+\frac{N_2-n_2}{n_2(N_2-1)}\right).
\end{split}
\end{equation}
Teraz pozostaje jedynie zastanowić się jak możemy wyliczyć parametr $p$. Czy to już oczywiste, że $p=(X_1+X_2)/(n_1+n_2)$, czy trzeba/można to jakoś pokazać?
\end{document}
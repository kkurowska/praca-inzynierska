\chapter{Analiza testów}
\label{r3}

W celu porównania testów opisanych w~rozdziale \ref{r2} napisano programy, które wyliczają prawdopodobieństwo błędu I~rodzaju i~moc testu oraz generują wykresy dla różnych parametrów.

\section{Porównanie testów ze skończoną poprawką}
\label{r3:skonczonetesty}

\subsection{Obszary krytyczne}
Najpierw porównano testy ze skończoną poprawką. Zacznijmy od wyznaczenia obszaru krytycznego dla testu~Z i~E na poziomie istotności $\alpha = 0.05$. Dla testu~Z w~podrozdziale~\ref{r2:testZ} został wyznaczony wzór (\ref{r2:ci}) na przedział ufności różnicy proporcji $p_1-p_2$. Wiadomo, że obszar krytyczny to dopełnienie przedziału ufności. Wobec tego wyliczając kwantyl~$z_{1-0.05/2}$ otrzymujemy, że zbiór wartości różnicy $p_1-p_2$, dla których test~Z odrzuca~$H_0$ jest równy
\begin{equation}
\left(-\infty,-1.96\sqrt{V_{X_1,X_2}}\right)\cup \left(1.96\sqrt{V_{X_1,X_2}}, \infty\right).
\end{equation}

\textit{Nie wiem jak konkretnie wyznaczyć te wartości, bo $V_{X_1,X_2}$ jest różne w zależności od $k_1$ i $k_2$}.

Dla testu~E nie jesteśmy w~stanie w~prosty sposób wyznaczyć obszaru krytycznego. Jednakże wiemy, że jest to zbiór wartości, dla których test~E odrzuca hipotezę zerową oraz będzie on postaci
\begin{equation}
(-\infty,a)\cup (b, \infty).
\end{equation}
Wobec czego możemy wyznaczyć wartości brzegowe $a$ i~$b$ dla konkretnych parametrów rozkładów $X_1$ i~$X_2$, sprawdzając kiedy test~E odrzuci $H_0$ dla każdych możliwych wartości $k_1$ i~$k_2$. Następnie, $a$ można wyliczyć jako największą ujemną wartość różnicy $p_1-p_2$, która została odrzucona, zaś $b$ jako najmniejszą dodatnią wartość $p_1-p_2$. W~tabeli \ref{r3:tabab} zostały przedstawione wyniki dla różnych parametrów, gdy $p_1=p_2$. Nie przedstawiono wyników w~przypadku $p_1\neq p_2$, ponieważ zmienienie jednej z~proporcji nie wpływa na zmianę wartości $a$ i~$b$.

\begin{table}
\centering
\caption{Wartości brzegowe obszaru krytycznego testu~E dla różnicy proporcji $p_1-p_2$, z różnymi parametrami rozkładów obserwacji}
\label{r3:tabab}
\begin{tabular}{|c|c|c|c|c|c|c|c|c|} \hline
	$n_1$ & $n_2$ & $M_1$ & $M_2$ & $N_1$ & $N_2$ & $p_1 = p_2$ & $a$ & $b$ \\
	\hline \hline
	5 & 5 & 50 & 50 & 100 & 100 & 0.5 & -0.6 & 0.6 \\
	10 & 10 & 30 & 30 & 100 & 100 & 0.3 & -0.3 & 0.3 \\
	5 & 20 & 10 & 10 & 100 & 100 & 0.1 & -0.5 & 0.35 \\
	15 & 15 & 18 & 30 & 30 & 50 & 0.6 & -0.2 & 0.2 \\
	40 & 40 & 1800 & 500 & 3000 & 1000 & 0.5 & -0.1 & 0.1 \\
	\hline
\end{tabular}
\end{table}

%Analizując wyniki przedstawione w~tabeli~\ref{r3:tabab} widzimy, że obszar krytyczny testu~E w~każdym z~przypadków jest mniejszy niż dla testu Z. W~związku z~czym test~E będzie rzadziej odrzucał hipotezę zerową.

\textit{Tu kilka wniosków... + w kolejnym podrozdziale odniesienie się do tych wniosków}

\subsection{Błąd I rodzaju i~moc testu}
Rysunek~\ref{sizeZE_n} na stronie~\pageref{sizeZE_n} przedstawia funkcję prawdopodobieństwa błędu I~rodzaju w~zależności od rozmiaru próbki dla różnych proporcji $p$. Po lewej stronie obserwacje z~obu populacji są tej samej wielkości $n_1=n_2=n$, natomiast po prawej się różnią. Pierwszym spostrzeżeniem jest to, że funkcja dla testu~E w~większości przypadków jest mniejsza niż dla testu Z. Dodatkowo dla testu~E prawdopodobieństwo błędu przekracza poziom istotności jedynie kilka razy, i~to bardzo nieznacznie. Tymczasem dla testu~Z funkcja częściej przyjmuje wartości powyżej $\alpha$, dochodząc nawet do $0.95$, jak na wykresie \ref{sizeZE_p_0_05}. Warto wspomnieć także, że dla większych $p$ (0.1 i~0.3) oraz wraz ze wzrostem wielkości próbki, funkcje są bliżej siebie i~bardziej skupiają się przy poziomie istotności $\alpha$.

\begin{figure}[p]
	\begin{subdiagrams}{sizeZE_p_0_05}{sizeZE_p_0_05_n1_5}
	\end{subdiagrams}
	
	\begin{subdiagrams}{sizeZE_p_0_1}{sizeZE_p_0_1_n1_10}
	\end{subdiagrams}
	
	\begin{subdiagrams}{sizeZE_p_0_3}{sizeZE_p_0_3_n1_5}
	\end{subdiagrams}
	\caption{Prawdopodobieństwo błędu I~rodzaju testów~Z i~E jako funkcja rozmiaru próbki $n$; $\alpha=0.05$; $N_1=N_2=100$}
	\label{sizeZE_n}
\end{figure}

Rysunek \ref{sizeZE_p} na stronie \pageref{sizeZE_p} obrazuje również prawdopodobieństwo błędu I~rodzaju, ale w~zależności od proporcji $p$. Analizując wykresy funkcji, można dojść do analogicznych wniosków, jak dla rysunku \ref{sizeZE_n}. W~każdym przypadku prawdopodobieństwo błędu I~rodzaju testu~Z jest większe od prawdopodobieństwa błędu testu E. Oprócz tego funkcja dla testu~E przekracza poziom istotności jedynie na wykresie \ref{sizeZE_n1_30_n2_15}, podczas gdy funkcja błędu testu~Z tylko w~jednym przypadku (\ref{sizeZE_n1_10_n2_6}) pozostaje całkowicie poniżej $\alpha$.

Podsumowując, dla małych rozmiarów próbek test~Z nie jest na poziomie istotności $\alpha$, co wynika z~zastosowania aproksymacji rozkładem normalnym do~wyliczenia $p$\dywiz wartości. Przybliżenie rozkładem normalnym działa dobrze tylko dla dużych prób. Wobec tego test~Z zbyt często odrzuca hipotezę zerową, gdy jest ona prawdziwa.
Pamiętajmy, że analizujemy testy dla małych populacji, zatem to że test~Z działa dla dużych prób nie jest wystarczające.

Rysunki \ref{powerZE_n1} i~\ref{powerZE_n2} na stronach \pageref{powerZE_n1} i~\pageref{powerZE_n2} zawierają wykresy mocy obu testów w~zależności od rozmiaru próbki dla różnych proporcji oraz rozmiarów populacji. Druga populacja ma ustaloną proporcję $p_2$, a~pierwsza przyjmuje trzy wartości różne od $p_2$. Dla obu testów zauważalny jest wzrost mocy wraz ze zwiększaniem się rozmiaru próbki. Ponadto moc testu wzrasta także dla proporcji bardziej oddalonych od siebie. W~każdym przypadku, dla próby $n=50$ i~różnicy między proporcjami $0.3$, moce mają wartość bliską $1$. 

Porównując, moc testu~Z jest większa od mocy testu E, jednakże różnice nie są znaczne. Ponadto, z~analizy błędu I~rodzaju, wiemy, że test~Z zbyt często odrzuca $H_0$, co wpływa na wyższą moc testu. Natomiast test~E utrzymuje poziom istotności $\alpha$, wobec czego niewiele mniejsza moc testu nie oznacza, że test~E jest gorszy. 

\begin{figure}[p]
	\begin{subdiagrams}{sizeZE_n_10}{sizeZE_n1_10_n2_6}
	\end{subdiagrams}
	
	\begin{subdiagrams}{sizeZE_n_20}{sizeZE_n1_20_n2_5}
	\end{subdiagrams}
	
	\begin{subdiagrams}{sizeZE_n_30}{sizeZE_n1_30_n2_15}
	\end{subdiagrams}
	\caption{Prawdopodobieństwo błędu I~rodzaju testów~Z i~E jako funkcja proporcji $p=M_1/N_1=M_2/N_2$; $\alpha=0.05$; $N_1=N_2=100$}
	\label{sizeZE_p}
\end{figure}


\begin{figure}[p]
	\begin{subdiagram}{powerZE_N1_30_N2_50_p_0_6}
	\end{subdiagram}
	\begin{subdiagram}{powerZE_N_100_p_0_1}
	\end{subdiagram}
	
	\caption{Moc testów~Z i~E jako funkcja rozmiaru próbki $n$, cz.~1}
	\label{powerZE_n1}
\end{figure}

\begin{figure}[p]
	\begin{subdiagram}{powerZE_N1_100_N2_200_p_0_1}
	\end{subdiagram}
	\begin{subdiagram}{powerZE_N1_3000_N2_100_p_0_6}
	\end{subdiagram}
	
	\caption{Moc testów~Z i~E jako funkcja rozmiaru próbki $n$, cz.~2}
	\label{powerZE_n2}
\end{figure}

\section{Porównanie testu bez skończonej poprawki~Zb z~testem E}

W niniejszym podrozdziale porównano test~Zb z~testem E. Nie włączono do analizy testu Z, z~powodu wniosków opisanych w~podrozdziale \ref{r3:skonczonetesty}. Poniższe rozważania zostały przeprowadzone, aby potwierdzić zasadność stosowania testu ze skończoną poprawką dla małych populacji lub o~stosunkowo dużej próbie.

Na rysunku \ref{sizeZbE_n} umieszczonym na stronie \pageref{sizeZbE_n} znajdują się wykresy prawdopodobieństwa błędu I~rodzaju analizowanych testów. Test E, jak stwierdzono w~podrozdziale \ref{r3:skonczonetesty}, nie przekracza poziomu istotności $\alpha$. Jednakże dla testu~Zb prawdopodobieństwo błędu I~rodzaju znacząco przekracza poziom istotności. Funkcja błędu~Zb przyjmuje wartości bliskie $0.2$, gdzie największa znajduje się na wykresie \ref{sizeZbE_p_0_1_n1_10}. Warto wspomnieć także, że funkcja jest największa w~przypadku małych próbek. Na wykresach \ref{sizeZbE_p_0_3} i~\ref{sizeZbE_p_0_3_n1_5} widać, że mniej więcej do $n=20$ test~Zb jest powyżej poziomu $\alpha$, następnie oscyluje, podobnie jak test E, w~okolicach poziomu istotności. 

Podsumowując, test~Zb nie jest adekwatny do rozważanego przypadku małej populacji gdy wartość $n$ jest mała. Test Zb odrzuca hipotezę zerową dużo częściej, niż powinien test na poziomie $\alpha$. Jest to spowodowane użyciem rozkładu dwumianowego w~formułowaniu statystyki testowej. Faktycznie dla dużych próbek różnica jest nieznaczna, natomiast w~sytuacji małej próbki rozbieżności są bardzo widoczne.

\begin{figure}[p]
	\begin{subdiagrams}{sizeZbE_p_0_05}{sizeZbE_p_0_05_n1_5}
	\end{subdiagrams}
	
	\begin{subdiagrams}{sizeZbE_p_0_1}{sizeZbE_p_0_1_n1_10}
	\end{subdiagrams}
	
	\begin{subdiagrams}{sizeZbE_p_0_3}{sizeZbE_p_0_3_n1_5}
	\end{subdiagrams}
	\caption{Prawdopodobieństwo błędu I~rodzaju testów~Zb i~E jako funkcja rozmiaru próbki $n$; $\alpha=0.05$; $N_1=N_2=100$}
	\label{sizeZbE_n}
\end{figure}

Rysunki \ref{powerZbE_n1} i \ref{powerZbE_n2} na stronach \pageref{powerZbE_n1} i \pageref{powerZbE_n2} przedstawiają wykresy mocy rozważanych testów w~zależności od rozmiaru próbki dla różnych rozmiarów populacji oraz proporcji $p$. Moc testu~Zb jest niewiele większa od mocy testu E. Jednakże podobnie jak w~przypadku testu Z, wynika to ze zbyt częstego odrzucania~$H_0$. Biorąc pod uwagę ten fakt, wyższa wartość testu~Zb nie świadczy o~jego poprawności.

\begin{figure}[p]
	\begin{subdiagram}{powerZbE_N1_30_N2_50_p_0_6}
	\end{subdiagram}
	\begin{subdiagram}{powerZbE_N_100_p_0_1}
	\end{subdiagram}

	\caption{Moc testów~Zb i~E jako funkcja rozmiaru próbki $n$, cz.~1}
	\label{powerZbE_n1}
\end{figure}

\begin{figure}[p]
	\begin{subdiagram}{powerZbE_N1_100_N2_200_p_0_1}
	\end{subdiagram}
	\begin{subdiagram}{powerZbE_N1_3000_N2_100_p_0_6}
	\end{subdiagram}
	
	\caption{Moc testów~Zb i~E jako funkcja rozmiaru próbki $n$, cz.~2}
	\label{powerZbE_n2}
\end{figure}

Ostatecznie pokazano, że test bez skończonej poprawki nie powinien być stosowany w~przypadku małej populacji, ponieważ wtedy znacznie przekracza poziom istotności $\alpha$. Najlepszym testem dla skończonej populacji okazał się test E, który dla różnych przypadków utrzymuje poziom istotności $\alpha$. 
\chapter*{Podsumowanie}

W pracy został rozważony problem testowania hipotez w~sytuacji populacji skończonego rozmiaru. W~rozdziale \ref{r1} przedstawione są schematy pobierania obserwacji. W~sytuacji nieskończonej populacji schemat opiera się o~rozkład dwumianowy i~próbka jest losowana ze zwracaniem. Natomiast, gdy populacja jest skończona, pobieranie obserwacji modeluje rozkład hipergeometryczny, przy czym elementy próbki są losowane bez zwracania. Po analizie obu rozkładów na konkretnym przykładzie sformułowano wnioski, że rozważane rozkłady dają podobne wyniki, jeśli populacja jest duża albo próbka stosunkowo mała. Jednakże, gdy obserwacja stanowi znaczną część populacji, albo sama populacja jest mała, to różnica między rozkładami staje się bardzo widoczna. Dzieje się tak, ponieważ rozkład hipergeometryczny uwzględnia wielkość populacji, a~rozkład Bernoulliego nie przechowuje takiej informacji.

Rozdział \ref{r3} zawiera analizę porównawczą testów opisanych w~rozdzialne \ref{r2} na podstawie wykresów prawdopodobieństwa błędu I rodzaju i~mocy testu dla różnych parametrów testowanych populacji. Porównanie dwóch testów ze skończoną poprawką wskazało na to, że test~Z nie utrzymuje poziomu istotności $\alpha$, ze względu na zastosowanie centralnego twierdzenia granicznego do statystyki testowej. Natomiast test~E nie wykracza znacząco powyżej poziomu istotności oraz jego moc jest niewiele mniejsza od mocy testu Z. Tym samym, test~E możemy uznać za dobry w~przypadku małej populacji.

W celu pokazania zasadności stosowania testu ze skończoną poprawką do małych populacji został także porównany test~E z~testem bez skończonej poprawki. Na wykresach przedstawiających błąd I rodzaju możemy zauważyć, że test~Zb znacząco przekracza poziom istotności $\alpha$, szczególnie dla małych próbek. Oznacza to, że zbyt często odrzuca hipotezę zerową. Wobec czego stosowanie testu bez skończonej poprawki w~sytuacji małej populacji wiąże się z~dużymi błędami.

Problem skończonej populacji często występuje w~medycynie. Lekarze badają przypadki, w~których populacja jest mała, ze względu na konkretne cechy badanych osób. We Wrocławiu w~Ośrodku Badawczo-Rozwojowym przy Wojewódzkim Szpitalu Specjalistycznym jeden z~kardiologów przeprowadza badania na populacji dzieci z~chorym sercem po operacji. Chce on sprawdzić jakie czynniki wpływają na wystąpienie powikłań pooperacyjnych u pacjentów. Niestety nie można na obecną chwilę przeprowadzić analizy danych, ponieważ badanie nie zostało jeszcze ukończone. Jednakże, aby sprawdzić, który czynnik jest najbardziej powiązany z~powikłaniami, można zastosować opisany w~pracy test E. Powinien dać on najlepsze wyniki spośród trzech testów przedstawionych w~pracy.
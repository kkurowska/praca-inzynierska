\documentclass[12pt]{mwrep}

\usepackage{polski}
\usepackage[utf8]{inputenc}
\usepackage{amsmath,amssymb,amsthm}
\usepackage{dsfont}

\newcommand{\1}[1]{\mathds{1}\left(#1\right)}


\begin{document}
Wiemy, że $p$-wartość możemy estymować jako
\begin{equation}
\sum_{x_1=L_{x_1}}^{U_{x_1}}\sum_{x_2=L_{x_2}}^{U_{x_2}} h(x_1;n_1,\hat{M_1},N_1)h(x_2;n_2,\hat{M_2},N_2) \1{|Z_{x_1,x_2}|\geq|Z_{k_1,k_2}|}
\end{equation}
i test E odrzuca $H_0$, gdy jest mniejsza $\alpha$. Więc jak byśmy chcieli wyznaczyć obszar krytyczny, to musielibyśmy jakoś wyciągnąć $p_1-p_2$ z tego
\begin{equation}
\begin{split}
\sum_{x_1=L_{x_1}}^{U_{x_1}}\sum_{x_2=L_{x_2}}^{U_{x_2}} h(x_1;n_1,\hat{M_1},N_1)h(x_2;n_2,\hat{M_2},N_2) \times \\
\times \1{-Z_{x_1,x_2}V_{k_1,k_2}\leq (p_1-p_2)\leq Z_{x_1,x_2}V_{k_1,k_2}} < \alpha.
\end{split}
\end{equation}
Ale nie wiem jak to można zrobić.
\end{document}
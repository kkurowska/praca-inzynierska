\chapter*{Wstęp}
Początki teorii rachunku prawdopodobieństwa i~statystyki sięgają XVI~w. Zajmowano się wtedy analizą rzutu kostką oraz prawdopodobieństwem błędów pomiarowych. Już w~XVII wieku Blaise Pascal sformułował i~dowiódł własności trójkąta arytmetycznego oraz użył pojęcia kombinacji~\cite{Hald2003}. Na początku XVIII~wieku opublikowane zostały prace Jacoba Bernoullego, w~których zawarł wiele swoich tez na temat prawdopodobieństwa. Przez te kilka wieków teoria rachunku prawdopodobieństwa i~statystyki znacząco się wzbogaciła i~rozwinęła. Rozpoczęto rozważania na temat estymacji i~testowania hipotez, które są w~naszych czasach zasadniczą domeną statystyki.

W przypadku dyskretnym najczęściej testowane są proporcje populacji~\cite{Lehmann1968}. Chcemy się przekonać czy dana próbka ma jakąś konkretną proporcję albo dwie próbki mają tę samą proporcję elementów z~badaną cechą. Znana jest powszechnie teoria dotycząca testowania hipotez, gdy populacja jest nieskończona, a~raczej na tyle duża, że możemy ją w przybliżeniu uznać za nieskończoną. Wtedy schemat próbkowania jest opisany jako losowanie ze zwracaniem. Jednak przypadek nieskończonej populacji nie wyczerpuje tematu testowania proporcji. Gdy populacja jest bardzo mała albo, gdy próbka jest niewiele mniejsza od całej populacji, schemat próbkowania opiera się o~losowanie bez zwracania. 

Warto zająć się teorią testowania hipotez dla skończonej populacji, ponieważ w~określonych przypadkach testy ze skończoną poprawką dają dużo dokładniejszą informację o~badanym przypadku niż testy zakładające nieskończoną populację. Ponadto zastosowanie tego typu testów ma duże znaczenie w~medycynie, gdzie często rozważane populacje mają na tyle wyspecjalizowane cechy, że są uważane za małe.

W~pierwszym rozdziale znajduje się opis schematu pobierania danych, w~przypadku nieskończonej i~skończonej populacji, oraz porównanie obu sposobów na podstawie własności rozkładów próbek i~estymację przedziałową proporcji. W rozdziale~\ref{r2} przedstawiono testy bez skończonej poprawki oraz z~jej uwzględnieniem. Rozdział~\ref{r3} zawiera porównanie testów na podstawie prawdopodobieństwa błędu I~rodzaju oraz mocy testu. Na końcu pracy znajduje się podsumowanie uzyskanych wyników.